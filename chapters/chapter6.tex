\chapter{บทสรุป}

\section{ปัญหาที่พบและการแก้ไข}
\subsection{การขาดปริมาณข้อมูล}
ทางคณะผู้จัดทำ พบปัญหากับปริมาณข้อมูลเรซูเมมีปริมาณที่ค่อนข้างน้อย\cite{dataset} 
เนื่องจากเป็นข้อมูลที่ละเอียดอ่อน และชุดข้อมูลยังค่อนข้างเก่าตั้งแต่ปีพุทธศักราช 2564 ทำให้ยากที่จะสามารถรวบรวมให้ได้ในปริมาณที่มากได้ ทำให้โมเดลยังรู้จักคำที่ไม่มาก และมักเป็นเครื่องมือที่ค่อนข้างเก่า
คณะผู้จัดทำจึงทำการแก้ไขปัญหานี้ด้วยการ
\begin{itemize}
    \item \textbf{รวบรวมข้อมูลเรซูเมจากผู้ใช้งานจริง} : คณะผู้จัดทำจะนำข้อมูลที่ผู้ใช้งานกรอกในการทำนายเรซูเมมาใช้เป็นชุดข้อมูลเพิ่มเติม โดยที่คณะผู้จัดทำ
จะขอความร่วมมือจากบุคคลที่ทำงานอยู่ในงานฝ่ายบุคคลของบริษัทเทคโนโลยีที่น่าเชื่อถือ หรืออาจารย์ภาควิชาวิศวกรรมคอมพิวเตอร์ มจธ. ในการระบุสายอาชีพจากเรซูเมนั้น
    \item \textbf{รวบรวมข้อมูลอื่นเพิ่มเติม} : คณะผู้จัดทำจะนำข้อมูลประกาศรับสมัครพนักงานของบริษัทเทคโนโลยีชั้นนำมารวบรวมเป็นข้อมูลสำหรับการปรับปรุงโมเดล
โดยที่คณะผู้จัดทำคาดหวังว่ารายละเอียดของการรับสมัครพนักงานเหล่านั้น จะทำให้โมเดลรู้จักคำสำคัญของแต่ละสายอาชีพ และช่วยให้การทำนายมีประสิทธิภาพมากยิ่งขึ้น
\end{itemize}
\subsection{ข้อจำกัดเรื่องการใช้งานเครื่องมือเพิ่มเติม ซึ่งต้องเสียค่าใช้จ่าย}
\par{
    ทางคณะผู้จัดทำ พบปัญหากับการใช้งานเครื่องมือเพิ่มเติมของโครงงาน นั่นคือ \hyperref[subsec:reactflow]{reactflow} ซึ่งไม่สามารถใช้งานเครื่องมือนี้ได้อย่างเต็มรูปแบบ
    เนื่องจากจำเป็นต้องเสียค่าใช้จ่ายจึงจะสามารถใช้งานได้ เช่น การจัดการตำแหน่งของหัวข้ออัตโนมัติ แอนิเมชันช่วยเหลือต่าง ๆ ดังนั้น หากต้องการใช้งานระบบเหล่านี้ จำเป็นจะต้องเขียนเครื่องมือทั้งหมดขึ้นมาเองใหม่ทั้งหมด
    ซึ่งหากใช้เวลาไปกับการพัฒนาเครื่องมือส่วนตรงนี้ อาจทำให้ไม่สามารถพัฒนาส่วนหลักสำหรับตอบโจทย์วัตถุประสงค์ได้สำเร็จ ทางเราจึงเลือกแก้ไขและพัฒนาเพียงส่วนที่จำเป็น และจำกัดการใช้งานแทน
    เช่น ผู้ใช้งานจะไม่สามารถเลือกและลากวัตถุภายในกราฟแสดงผลได้เอง และต้องใช้งานวิธีการจัดตำแหน่งของเราเท่านั้น
}
\subsection{เครื่องการพัฒนาส่วน front-end (NextJS) ไม่สามารถเข้าถึงตัวแปร environment ในระบบ Cloud ได้โดยตรง}
\par{
    เนื่องจากเวอร์ชันของ NextJs ที่ทางคณะจัดทำใช้งานในปัจจุบัน ไม่สามารถขอเข้าถึงตัวแปร environment ภายใน Google Cloud ซึ่งทางคณะผู้จัดทำใช้สำหรับการปล่อยให้ใช้งานได้โดยตรง
    ทางเราจึงเลือกแก้ปัญหาด้วยการนำตัวแปรเหล่านั้น ไปเก็บไว้ที่บริการอื่น ซึ่งคือ GitHub Secrets จากนั้นจึงสร้าง workflow สำหรับการนำไฟล์โค้ดทั้งหมดของโครงงาน
    ไป build พร้อมตัวแปรใน GitHub Secrets และขออนุญาต deploy ไปที่ Google Cloud แทน ซึ่งแก้ไขได้สำเร็จแล้วและใช้งานอยู่จริง
}

\section{แนวทางการพัฒนาต่อไปในอนาคต}
\subsection{Web Scraping เพื่อการแนะนำช่องทางการเรียนรู้}
\par{
    ทางคณะผู้จัดทำได้เล็งเห็นว่า หากต้องการช่วยแนะนำช่องทางการเรียนรู้ การนำข้อมูลภายนอกมาช่วยนำเสนอภายในเว็บแอปพลิเคชันก็เป็นเรื่องที่ดี และมีความต้องการจากผู้ใช้กล่าวถึงมาบ้างเช่นกันในกระบวนการเก็บข้อมูลที่ทางเราเคยดำเนินการไป
    โดยทางเราเพ่งเล็งจะนำเสนอข้อมูลจำพวก คอร์สเรียน แหล่งเรียนรู้ ซึ่งสามารถนำไปพัฒนาด้วยตนเองต่อได้ อย่างไรก็ตาม สาเหตุที่ฟีเจอร์นี้ไม่สามารถนำมาพัฒนาได้ภายในระยะเวลาที่กำหนด
    เป็นเพราะเครื่องสำหรับช่วยเหลือการดึงข้อมูล (scraping) นั้น จำเป็นต้องเสียค่าใช้จ่ายในการใช้บริการขอดึงข้อมูล จึงทำให้ทางเรามีข้อจำกัดและไม่สามารถใช้งานได้โดยตรง แต่ก็ยังคงเป็นฟีเจอร์ที่ควรค่าแก่การพัฒนา ดังนั้น
    ทางคณะผู้จัดทำจึงสนใจพัฒนาระบบ web-scraping เพื่อการแนะนำช่องทางการเรียนรู้เป็นอย่างมากหากมีโอกาสในอนาคต
}
\subsection{Optical Character Recognition (OCR) เพื่อเสริมประสบการณ์การกรอกข้อมูลเรซูเม}
\label{subsec:Future Plan OCR}
\par{
    จากผลตอบรับภายหลังการทดสอบกับผู้ใช้ เราได้เล็งเห็นแล้วว่า ผู้ใช้งานไม่ต้องการกรอกข้อมูลเองทั้งหมด และหากเป็นไปได้ ก็อยากได้ตัวช่วยช่วยเหลือการกรอกข้อมูล
    เช่น การอัปโหลดไฟล์ pdf ขึ้นสู่ระบบโดยตรงเพื่อกรอกให้เองอัตโนมัติ อย่างไรก็ตาม สาเหตุที่ระบบนี้ไม่อยู่ในขอบเขตการทำตั้งแต่เริ่มต้นนั้น เป็นเพราะอุปสรรคหลายอย่าง
    เช่น ผู้ใช้งานไม่ได้ใช้วิธีการเขียนเรซูเมที่ข้อมูลเรียงต่อกันเป็นสัดส่วนที่ชัดเจน มีการใช้รูปแทนการใช้ข้อความ ส่งผลให้ไม่สามารถนำข้อมูลอักษรที่ได้รับภายหลังการ OCR มาใช้งานต่อได้โดยตรง
    ดังนั้น หากต้องการพัฒนาระบบนี้อย่างจริงจัง จึงต้องตระเตรียมเรื่องของวิธีการนำเข้าข้อมูลทุกอย่างภายในเรซูเมมาใช้ให้ได้ และหาวิธีการนำข้อมูลเหล่านั้นมาแบ่งส่วนให้ชัดเจนว่าส่วนใดคือทักษะ ประสบการณ์ หรือการศึกษา รวมไปถึงข้อมูลส่วนตัวที่ต้องนำไปคัดออกในท้ายที่สุดเพื่อความเป็นส่วนตัว
    ซึ่งอาจต้องทำการเทรนหรือพัฒนาโมเดลอื่น ๆ มาเพื่อใช้งานร่วมกันกับระบบนี้ สิ่งนี้จึงเป็นฟีเจอร์ที่ต้องลงแรงและเวลาสูง แต่ผู้ใช้เองก็มีความต้องการเช่นกัน จึงเหมาะแก่การนำมาเป็นแผนการในอนาคตที่น่าสนใจอย่างมาก
}

\subsection{การเทรนโมเดลเพิ่มเติม}
\par{
    แม้ในโครงงานนี้ ทางคณะผู้จัดจะทำการเทรนโมเดลจนสำเร็จตามเป้าหมายแล้ว อย่างไรก็ตาม เรามองว่าโมเดลนี้ยังสามารถพัฒนาได้อีก เพราะในปัจจุบันพวกเราพบกับปัญหาเชิงเทคนิคที่สามารถแก้ไขได้หากมีเวลาเพิ่มเติม
    เช่น ปริมาณชุดข้อมูลที่นำมาใช้ในการเทรน ซึ่งภายหลังการทดสอบโครงงาน เราได้รับข้อมูลเรซูเมเพิ่มเติมมาจำนวนมาก และอาจเพิ่มได้อีกหากเปิดให้ใช้งานต่อไป
    ดังนั้น เราสามารถใช้ประโยชน์จากจุดนี้ ในการรวบรวมข้อมูลจากฐานข้อมูลปัจจุบัน นำมาคัดกรองข้อมูล นิยามประเภทของเรซูเมผ่านการปรึกษาผู้เชี่ยวชาญ 
    และเทรนโมเดลใหม่อีกครั้งได้ ซึ่งการเพิ่มเติมประสิทธิภาพนี้ อาจสามารถช่วยให้มีชุดข้อมูลเพิ่มขึ้น ประสิทธิภาพดีขึ้น หรืออาจมีสายอาชีพใหม่ ๆ ที่สนับสนุนเพิ่มได้
    จึงน่าสนใจเป็นอย่างมากหากมีโอกาสในการพัฒนาอีกครั้งในอนาคต
}

\section{สรุปผลโครงงาน}
\par{
    โครงงาน Compath เว็บแอปพลิเคชันแยกประเภทเรซูเมสำหรับนักศึกษาวิศวกรรมคอมพิวเตอร์เพื่อแนะนำสายอาชีพ
    ถูกสร้างขึ้นมาเพื่อจุดประสงค์ในการช่วยปรับปรุงเรซูเมของนักศึกษาวิศวกรรมคอมพิวเตอร์ มจธ. เสริมสร้างความมั่นใจ เห็นแนวโน้มของเรซูเมตนเอง เพื่อนำไปปรับปรุงให้เป็นไปตามต้องการและได้รับแนวทางการพัฒนาตนเอง
    โดยการใช้งานร่วมกับปัญญาประดิษฐ์ประเภท Machine Learning โมเดล KNN อันมีผลลัพธ์ที่ดีที่สุดมาช่วยคัดแยกประเภทเรซูเม ซึ่งมีความแม่นยำที่ 78.57\% ถือว่าสำเร็จตามเป้าหมาย
    
    โดยจากการทดสอบกับผู้ใช้ พบว่า ได้รับผลตอบรับเฉลี่ยอยู่ที่ 4.47 คะแนน จากคะแนนเต็ม 5 ซึ่งนับเป็นร้อยละ 89 โดยทดสอบกับผู้ใช้จำนวน 70 คน ซึ่งเป็นผลตอบรับที่น่าพึงพอใจ และถือว่าประสบความสำเร็จ
}
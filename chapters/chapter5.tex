\chapter{บทสรุป}

\section{สรุปผลเบื้องต้นของโครงงาน}
\par{ภาพรวมของโครงงาน...}

\section{ปัญหาที่พบและการแก้ไข}
\subsection{การขาดปริมาณข้อมูล}
ทางคณะผู้จัดทำ พบปัญหากับปริมาณข้อมูลเรซูเมมีปริมาณที่ค่อนข้างน้อย\cite{dataset} 
เนื่องจากเป็นข้อมูลที่ละเอียดอ่อน และชุดข้อมูลยังค่อนข้างเก่าตั้งแต่ปีพุทธศักราช 2564 ทำให้ยากที่จะสามารถรวบรวมให้ได้ในปริมาณที่มากได้ ทำให้โมเดลยังรู้จักคำที่ไม่มาก และมักเป็นเครื่องมือที่ค่อนข้างเก่า
คณะผู้จัดทำจึงทำการแก้ไขปัญหานี้ด้วยการ
\begin{itemize}
    \item \textbf{รวบรวมข้อมูลเรซูเมจากผู้ใช้งานจริง} : คณะผู้จัดทำจะนำข้อมูลที่ผู้ใช้งานกรอกในการทำนายเรซูเมมาใช้เป็นชุดข้อมูลเพิ่มเติม โดยที่คณะผู้จัดทำ
จะขอความร่วมมือจากบุคคลที่ทำงานอยู่ในงานฝ่ายบุคคลของบริษัทเทคโนโลยีที่น่าเชื่อถือ หรืออาจารย์ภาควิชาวิศวกรรมคอมพิวเตอร์ มจธ. ในการระบุสายอาชีพจากเรซูเมนั้น
    \item \textbf{รวบรวมข้อมูลอื่นเพิ่มเติม} : คณะผู้จัดทำจะนำข้อมูลประกาศรับสมัครพนักงานของบริษัทเทคโนโลยีชั้นนำมารวบรวมเป็นข้อมูลสำหรับการปรับปรุงโมเดล
โดยที่คณะผู้จัดทำคาดหวังว่ารายละเอียดของการรับสมัครพนักงานเหล่านั้น จะทำให้โมเดลรู้จักคำสำคัญของแต่ละสายอาชีพ และช่วยให้การทำนายมีประสิทธิภาพมากยิ่งขึ้น
\end{itemize}
\subsection{ข้อจำกัดเรื่องการใช้งานเครื่องมือเพิ่มเติม ซึ่งต้องเสียค่าใช้จ่าย}
\par{แนวทางการแก้ไข....}

\section{แนวทางการพัฒนาต่อไปในอนาคต}
\subsection{Web Scraping เพื่อการแนะนำช่องทางการเรียนรู้}
\par{แนวทางการพัฒนา จะนำมาใช้อย่างไร เพื่ออะไร....}
\chapter{วิธีการดำเนินงาน}

\emph{หัวข้อต่าง ๆ ในแต่ละบทเป็นเพียงตัวอย่างเท่านั้น หัวข้อที่จะใส่ในแต่ละบทขึ้นอยู่กับโปรเจคของนักศึกษาและอาจารย์ที่ปรึกษา}


ตัวอย่างการใส่อ้างอิงที่มา -> \cite{hypersense} ถ้าต้องการใส่แหล่งอ้างอิงมากกว่า 1 ให้ทำดังนี้ -> \cite{hypersense,bworld}
Explain the design (how you plan to implement your work) of your project. Adjust the section titles below to suit the types of your work. Detailed physical design like circuits and source codes should be placed in the appendix.

\section{ข้อกำหนดและความต้องการของระบบ}

\section{สถาปัตยกรรมระบบ}

\begin{table}[!h]
    \centering
    \caption{test table x1}\label{tbl:symbols}
    \begin{tabular}{@{}p{0.07\textwidth}|p{0.7\textwidth}p{0.1\textwidth}}\hline
        \multicolumn{2}{l}{\textbf{SYMBOL}} & \textbf{UNIT}                         \\ \hline
        $\alpha$                            & Test variable\hfill     & m$^2$       \\
        $\lambda$                           & Interarrival rate\hfill & jobs/second \\
        $\mu$                               & Service rate\hfill      & jobs/second \\ \hline
    \end{tabular}
    %\begin{tabular}{c|c} \hline
    % $\alpha$ & $\beta$ \\ \hline
    % $\delta$ & $\mu$ \\ \hline
    %\end{tabular}
\end{table}



\section{Hardware Module 1}
\subsection{Component 1}
\subsection{Logical Circuit Diagram}

\section{Hardware Module 2}
\subsection{Component 1}
\subsection{Component 2}

\section{Path Finding Algorithm}

\section{Database Design}

\section{UML Design}

\section{GUI Design}

\section{การออกแบบการทดลอง}
\subsection{ตัวชี้วัดและปัจจัยที่ศึกษา}
\subsection{รูปแบบการเก็บข้อมูล}
\documentclass[12pt,oneside,openright,a4paper]{cpe-thai-project}
\usepackage{polyglossia}
\usepackage{indentfirst}
\usepackage{float}
\usepackage{hyperref}
\usepackage{multirow}
\usepackage{tabularx}
\usepackage{longtable}
\usepackage[table]{xcolor}

\setmainfont[
    Path = fonts/,
    Extension = .ttf ,
    ItalicFont = * Italic ,
    BoldFont = * Bold ,
    BoldItalicFont = * BoldItalic ,
    Script = Thai,
    Mapping=tex-text,
    Scale=1.23,
    LetterSpace=0.0,
    FakeStretch=1.0]{THSarabunNew}
\setlength\parindent{24pt}
\setdefaultlanguage{thai}
\setotherlanguage{english}
\XeTeXlinebreaklocale "th"
\XeTeXlinebreakskip=0pt plus 0pt
\emergencystretch=10pt

% First line of title
\def\disstitleone{Compath: เว็บแอปพลิเคชันแยกประเภทเรซูเมสำหรับนักศึกษาวิศวกรรมคอมพิวเตอร์เพื่อแนะนำสายอาชีพ}   
% Second line of title
\def\disstitletwo{Compath: Web application with resume classification for CPE students to recommend their career path}   

% author section
\def\dissauthor{63070501025 Niwatchai Wangtrakuldee}
\def\dissauthortwo{63070501038 Napat Wareedee}
\def\dissauthorthree{63070501039 Narith Thanomsup}

\def\dissdegree{Bachelor of Engineering} % Name of the degree
\def\dissdegreeabrev{B.Eng} % Abbreviation of the degree
\def\dissyear{2023}                   % Year of submission
\def\thaidissyear{2566}               % Year of submission (B.E.)

%%%%%%%%%%%%%%%%%%%%%%%%%%%%%%%%%%%%%%%%%%%%
% Your project and independent study committee..
%%%%%%%%%%%%%%%%%%%%%%%%%%%%%%%%%%%%%%%%%%%%
\def\dissadvisor{Asst.Prof.Dr. Suthathip Maneewongvatana, Ph.D.}  % Advisor
\def\disscoadvisor{}  % Co-advisor

\def\disscommitteetwo{Asst.Prof.Dr. Phond Phunchongharn, Ph.D.}  % 3rd committee member (optional)
\def\disscommitteethree{Asst.Prof.Dr. Priyakorn Pusawiro, Ph.D.}   % 4th committee member (optional) 
\def\disscommitteefour{Assoc.Prof.Dr Thumrongrat Amornraksa, Ph.D.}    % 5th committee member (optional) 

\def\worktype{Project} %%  Project or Independent study
\def\disscredit{3}   %% 3 credits or 6 credits


\def\fieldofstudy{Computer Engineering} 
\def\department{Computer Engineering} 
\def\faculty{Engineering}

\def\thaifieldofstudy{วิศวกรรมคอมพิวเตอร์} 
\def\thaidepartment{วิศวกรรมคอมพิวเตอร์} 
\def\thaifaculty{วิศวกรรมศาสตร์}
 
\def\appendixnames{Appendix} %%% Appendices or Appendix

\def\thaiworktype{ปริญญานิพนธ์} %  Project or research project % 
\def\thaidisstitleone{Compath: เว็บแอปพลิเคชันแยกประเภทเรซูเมสำหรับนักศึกษาวิศวกรรมคอมพิวเตอร์เพื่อแนะนำสายอาชีพ}
\def\thaidisstitletwo{Compath: Web application with resume classification for CPE students to recommend their career path}
\def\thaidissauthor{63070501025 นายณิวัฒน์ชัย หวังตระกูลดี}
\def\thaidissauthortwo{63070501038 นายนภัทร วารีดี} %Optional
\def\thaidissauthorthree{63070501039 นายนริศ ถนอมทรัพย์} %Optional

\def\thaidissadvisor{ผศ. ดร.สุธาทิพย์ มณีวงศ์วัฒนา}
%% Leave this empty if you have no co-advisor
\def\thaidisscoadvisor{} %Optional
\def\thaidissdegree{วิศวกรรมศาสตรบัณฑิต}

% Change the line spacing here...
\linespread{1.15}

%%%%%%%%%%%%%%%%%%%%%%%%%%%%%%%%%%%%%%%%%%%%%%%%%%%%%%%%%%%%%%%%
% End of personal customization.  Do not modify from this part 
% to \begin{document} unless you know what you are doing...
%%%%%%%%%%%%%%%%%%%%%%%%%%%%%%%%%%%%%%%%%%%%%%%%%%%%%%%%%%%%%%%%


%%%%%%%%%%%% Dissertation style %%%%%%%%%%%
%\linespread{1.6} % Double-spaced  
%%\oddsidemargin    0.5in
%%\evensidemargin   0.5in
%%%%%%%%%%%%%%%%%%%%%%%%%%%%%%%%%%%%%%%%%%%
%\renewcommand{\subfigtopskip}{10pt}
%\renewcommand{\subfigbottomskip}{-5pt} 
%\renewcommand{\subfigcapskip}{-6pt} %vertical space between caption
%                                    %and figure.
%\renewcommand{\subfigcapmargin}{0pt}

\renewcommand{\topfraction}{0.85}
\renewcommand{\textfraction}{0.1}

\newtheorem{theorem}{Theorem}
\newtheorem{lemma}{Lemma}
\newtheorem{corollary}{Corollary}

\def\QED{\mbox{\rule[0pt]{1.5ex}{1.5ex}}}
\def\proof{\noindent\hspace{2em}{\itshape Proof: }}
\def\endproof{\hspace*{\fill}~\QED\par\endtrivlist\unskip}
%\newenvironment{proof}{{\sc Proof:}}{~\hfill \blacksquare}
%% The hyperref package redefines the \appendix. This one 
%% is from the dissertation.cls
%\def\appendix#1{\iffirstappendix \appendixcover \firstappendixfalse \fi \chapter{#1}}
%\renewcommand{\arraystretch}{0.8}
%%%%%%%%%%%%%%%%%%%%%%%%%%%%%%%%%%%%%%%%%%%%%%%%%%%%%%%%%%%%%%%%
%%%%%%%%%%%%%%%%%%%%%%%%%%%%%%%%%%%%%%%%%%%%%%%%%%%%%%%%%%%%%%%%

\usepackage{ragged2e}
\begin{document}

\pdfstringdefDisableCommands{%
    \let\MakeUppercase\relax
}

\begin{center}
    \includegraphics[width=2.8cm]{logo02.jpg}
\end{center}
\vspace*{-1cm}

\maketitlepage
\makesignaturepage

%%%%%%%%%%%%%%%%%%%%%%%%%%%%%%%%%%%%%%%%%%%%%%%%%%%%%%%%%%%%%%
%%%%%%%%%%%%%%%%%%%%%% English abstract %%%%%%%%%%%%%%%%%%%%%%%
%%%%%%%%%%%%%%%%%%%%%%%%%%%%%%%%%%%%%%%%%%%%%%%%%%%%%%%%%%%%%%
\abstract

The motivation of this project is from the problem that students do not know how to develop knowledge and skills required to 
employ their desired career paths. Sometimes they may not even know what career they want or are suited for. This affects when 
they are writing a resume, that do not reflect the necessary knowledge for their desired job when applying.

To solve this problem, we aimed to develop a web application along with machine learning model to classification career paths from 
text data in the target group's resumes to help them modify resume and develop themselves. In the model training process, we used a raw dataset of 962 records.
After consolidating related careers into 6 groups, removing duplicates and low-quality records, 205 records remained 
for training, and 14 additional records that we collected more by ourselves for quality testing. We trained and developed multiple 
models for testing our hypothesis, using the TF-IDF algorithm to identify significant words inside resume based on words frequency.

At the end of development, the best performing model was K-Nearest Neighbors (KNN), with 78.57\% accuracy.
We also get a result from web application usability testing for evaluate this project as average score at 4.47 out of 5,
with 70 peoples in target group, that is 89\% positive feedback, considered as a succeed, and which is this project 
was able to successfully fulfill its purpose.

\begin{flushleft}
    \begin{tabular*}{\textwidth}{@{}lp{0.8\textwidth}}
        \textbf{Keywords}: & Web Application, Machine Learning, Deep Learning, Resume
    \end{tabular*}
\end{flushleft}
\endabstract

%%%%%%%%%%%%%%%%%%%%%%%%%%%%%%%%%%%%%%%%%%%%%%%%%%%%%%%%%%%%%%
%%%%%%%%%% Thai abstract here %%%%%%%%%%%%%%%%%%%%%%%%%%%%%%%%%
%%%%%%%%%%%%%%%%%%%%%%%%%%%%%%%%%%%%%%%%%%%%%%%%%%%%%%%%%%%%%%
% {\newfontfamily\thaifont{TH Sarabun New:script=thai}[Scale=1.3]
% \XeTeXlinebreaklocale "th_TH"	
% \thaifont
\thaiabstract

จุดเริ่มต้นของโครงงานนี้มาจากปัญหาที่นักศึกษาไม่ทราบว่าควรพัฒนาความรู้และทักษะใด เพื่อให้ตนเองเหมาะสมกับสายอาชีพที่สนใจ 
นอกจากนี้บางครั้งผู้เรียนอาจไม่ทราบสายอาชีพที่ตนเองต้องการหรือถนัด ซึ่งส่งผลต่อเนื่องไปถึงการเขียนประวัติย่อ (resume) 
ที่ไม่ได้สะท้อนถึงองค์ความรู้ที่จำเป็นต่อสายอาชีพอย่างที่ต้องการในการสมัครงาน

เราจึงคิดพัฒนาเว็บแอปพลิเคชันพร้อมกับการเรียนรู้ของเครื่องด้วยด้วยปัญญาประดิษฐ์ เพื่อนำมาแยกประเภทสายอาชีพจากข้อมูลประวัติย่อของกลุ่มเป้าหมาย 
เพื่อช่วยปรับปรุงการเขียนเรซูเมที่ดีขึ้นและหาแนวทางการพัฒนาตนเอง ในการฝึกสอนโมเดลปัญญาประดิษฐ์ เราได้นำชุดข้อมูลเรซูเมดิบจำนวน 962 รายการมาใช้ 
จากนั้นนำข้อมูลเหล่านั้นมาคัดกรอง ซึ่งเราได้ทำการรวมสายอาชีพที่เกี่ยวข้องกันให้เหลือเพียง 6 สายอาชีพ คัดกรองรายการที่มีความซ้ำกันและไม่มีคุณภาพ 
จนเหลือเพียง 205 รายการที่เหมาะสมแก่การใช้งาน และ 14 รายการสำหรับการทดสอบคุณภาพซึ่งทางเราเป็นผู้รวบรวมมาด้วยตัวเอง 
โดยเรานำชุดข้อมูลเหล่านี้ไปฝึกสอนและพัฒนากับโมเดลหลายประเภทเพื่อทดสอบสมมติฐาน และใช้งานอัลกอริทึม TF-IDF 
กับโมเดลแต่ละประเภทเพื่อจับจุดคำสำคัญของเรซูเมจากความถี่ของคำภายใน

หลังเสร็จสิ้นกระบวนการพัฒนา โมเดลที่ให้ผลลัพธ์ที่ดีที่สุดคือ K-Nearest Neighbors (KNN) ซึ่งมีความแม่นยำที่ 78.57\% 
และผลตอบรับของเว็บแอปพลิเคชัน ได้รับคะแนนเฉลี่ยอยู่ที่ 4.47 เต็ม 5 จากผู้ทดสอบจำนวน 70 คน 
นับเป็นคะแนนเชิงบวก 89\% ซึ่งถือว่าบรรลุเป้าหมายที่วางไว้ และนับว่าโครงงานนี้สามารถทำตามจุดประสงค์ได้สำเร็จ

\begin{flushleft}
    \begin{tabular*}{\textwidth}{@{}lp{0.8\textwidth}}
        & \\

        \textbf{คำสำคัญ}: & เว็บแอปพลิเคชัน, การเรียนรู้ของเครื่อง, การเรียนรู้เชิงลึก, เรซูเม
    \end{tabular*}
\end{flushleft}
\endabstract

%}

%%%%%%%%%%%%%%%%%%%%%%%%%%%%%%%%%%%%%%%%%%%%%%%%%%%%%%%%%%%%
%%%%%%%%%%%%%%%%%%%%%%% Acknowledgments %%%%%%%%%%%%%%%%%%%%
%%%%%%%%%%%%%%%%%%%%%%%%%%%%%%%%%%%%%%%%%%%%%%%%%%%%%%%%%%%%
\preface
ขอขอบพระคุณอาจารย์ที่ปรึกษา คณะกรรมการ พ่อแม่พี่น้อง และผศ.ดร.ขจรพงษ์ อัครจิตสกุล ผู้ซึ่งเป็นอาจารย์ที่ปรึกษาแรกเริ่มในภาคการศึกษาที่ 1
รวมไปถึงผู้ให้อนุญาตในการทดสอบโครงงานทุกท่าน
ที่ช่วยให้โครงงานนี้สำเร็จไปได้ด้วยดี และถูกปรับปรุงจนสามารถนำมาใช้งานได้ตามที่คาดหวังไว้ และมีประสิทธิภาพมากพอในระดับที่น่าพึงพอใจ

และขอขอบคุณแก่ผู้ให้ความช่วยเหลือทางอ้อม นั่นคือ นายศุภกร รักนะ และเพื่อนร่วมชั้นปีคนอื่น ๆ ที่ให้ความร่วมมือในการถ่ายวิดีโอโปรโมตต่าง ๆ เป็นนักแสดงที่มากความสามารถ
และให้ความเห็นที่ดีในการปรับปรุงพัฒนาโครงงานเรื่อยมา

%%%%%%%%%%%%%%%%%%%%%%%%%%%%%%%%%%%%%%%%%%%%%%%%%%%%%%%%%%%%%
%%%%%%%%%%%%%%%% ToC, List of figures/tables %%%%%%%%%%%%%%%%
%%%%%%%%%%%%%%%%%%%%%%%%%%%%%%%%%%%%%%%%%%%%%%%%%%%%%%%%%%%%%
% The three commands below automatically generate the table 
% of content, list of tables and list of figures
\tableofcontents
\listoftables
\listoffigures

%%%%%%%%%%%%%%%%%%%%%%%%%%%%%%%%%%%%%%%%%%%%%%%%%%%%%%%%%%%%%%
%%%%%%%%%%%%%%%%%%%%% List of symbols page %%%%%%%%%%%%%%%%%%%
%%%%%%%%%%%%%%%%%%%%%%%%%%%%%%%%%%%%%%%%%%%%%%%%%%%%%%%%%%%%%%
% You have to add this manually..
% \listofsymbols
% \begin{flushleft}
%     \begin{tabular}{@{}p{0.07\textwidth}p{0.7\textwidth}p{0.1\textwidth}}
%         \textbf{SYMBOL} &                        & \textbf{UNIT} \\[0.2cm]
%         $\alpha$        & Test variable\hfill    & m$^2$         \\
%         $\lambda$       & Interarival rate\hfill & jobs/second   \\
%         $\mu$           & Service rate\hfill     & jobs/second   \\
%     \end{tabular}
% \end{flushleft}
%%%%%%%%%%%%%%%%%%%%%%%%%%%%%%%%%%%%%%%%%%%%%%%%%%%%%%%%%%%%%%
%%%%%%%%%%%%%%%%%%%%% List of vocabs & terms %%%%%%%%%%%%%%%%%
%%%%%%%%%%%%%%%%%%%%%%%%%%%%%%%%%%%%%%%%%%%%%%%%%%%%%%%%%%%%%%
% You also have to add this manually..
\listofvocab
\begin{flushleft}
    \begin{tabular}{@{}p{1in}@{=\extracolsep{0.5in}}p{0.73\textwidth}}
        เรซูเม & ประวัติย่อสำหรับใช้ในการสมัครงานต่าง ๆ (resume) \\
        เทรน  & การฝึกสอนโมเดล (train model) ปัญญาประดิษฐ์ \\
        deploy & การนำบริการต่าง ๆ เช่น เว็บแอปพลิเคชันหรือ server ขึ้นสู่บริการ cloud และสามารถเข้าถึงได้แบบสาธารณะ \\
    \end{tabular}
\end{flushleft}

\setlength{\parskip}{1.2mm}

%%%%%%%%%%%%%%%%%%%%%%%%%%%%%%%%%%%%%%%%%%%%%%%%%%%%%%%%%%%%%%%
%%%%%%%%%%%%%%%%%%%%%%%% Main body %%%%%%%%%%%%%%%%%%%%%%%%%%%%
%%%%%%%%%%%%%%%%%%%%%%%%%%%%%%%%%%%%%%%%%%%%%%%%%%%%%%%%%%%%%%%


\chapter{บทนำ}

\section{ที่มาและความสำคัญ}


ในช่วงเวลาปัจจุบันนี้ เป็นช่วงที่เกิดปัญหานักศึกษาจบใหม่ว่างงานเพิ่มขึ้นทุกปี \cite{mrgonline} และมีแนวโน้มจะเพิ่มขึ้นอีกเรื่อย ๆ ซึ่งอาจส่งผลให้เกิดความเครียดในหมูนักศึกษาจบใหม่ กระทบการวางแผนชีวิตในอนาคต อาจต้องมีการย้ายสายงาน ทำงานไม่ตรงสายการเรียน เป็นต้น โดยปัญหานี้ เกิดมาจากปัจจัยหลายอย่างทั้งในแง่ระบบการปกครอง ความต้องการของสายอาชีพต่าง ๆ ในตลาดแรงงานที่เปลี่ยนแปลงไป ระบบการศึกษา หรืออื่น ๆ อีกมากมายเกินที่เราจะควบคุมได้ อย่างไรก็ตาม หากเป็นการส่งเสริมด้านการศึกษาเพิ่มเติมด้วยตัวเองและชี้แนะแนวทางนั้น สามารถเป็นไปได้ ทางคณะผู้จัดทำจึงได้เริ่มมองหาจุดที่สามารถเข้าช่วยเหลือและบรรเทาปัญหานี้ โดยเริ่มตั้งเป้าหมายไว้ที่นักศึกษามหาวิทยาลัยเทคโนโลโลยีพระจอมเกล้าธนบุรี คณะวิศวกรรมศาสตร์ สาขาวิศวกรรมคอมพิวเตอร์เป็นกลุ่มแรกก่อน เพื่อนำมาพิสูจน์ผลลัพธ์ของวิธีแก้ปัญหาที่เราออกแบบ

จากการสำรวจกับกลุ่มเป้าหมาย ทางคณะผู้จัดทำได้เล็งเห็นถึงปัญหาร่วมบางอย่าง ซึ่งทางคณะผู้จัดทำเองก็ได้ประสบพอเจอด้วยตัวเองตั้งแต่ช่วงชั้นปีที่ 1 และยังคงพบเจออยู่จนถึงปัจจุบันเช่นกัน นั่นคือปัญหาในการค้นหาและรับรู้สายอาชีพที่เหมาะสมกับตนเอง แนวทางการเรียนรู้และพัฒนาความสามารถ ไม่ทราบศาสตร์ความรู้ที่จำเป็นต่อการไปสู่สายงานนั้น ๆ เช่น วิชาเรียนที่ควรเลือกในช่วงมหาวิทยาลัย ค่ายหรืองานแข่งพัฒนาทักษะต่าง ๆ ที่ควรเข้าร่วมเพื่อเสริมประสบการณ์ แม้กระทั่งรายละเอียดหรือหน้าที่ความรับผิดชอบของสายอาชีพต่าง ๆ ก็ยังไม่เป็นที่เข้าใจอย่างถูกต้องในหมู่นักศึกษาหลายคน โดยพบว่ามีผู้ที่ประสบปัญหานี้ยังคงมีอยู่ทุกชั้นปี

จากสิ่งที่กล่าวไป ส่งผลให้นักศึกษาหลายคนไม่สามารถตอบได้ว่าตนเองควรพัฒนาตนเองอย่างไร สายอาชีพใดคือสิ่งที่ใช่สำหรับตนเอง แม้จะรู้ว่าต้องการไปในสายงานใดก็ไม่อาจทราบได้ว่าต้องพัฒนาตนเองอย่างไรต่อไปจึงจะตอบโจทย์ตลาดแรงงาน หรือรู้ตัวช้าเกินไปจนพัฒนาได้ไม่ทันการ ซึ่งสะท้อนปัญหาที่ทางคณะผู้จัดเองก็ได้พบเจอด้วย

หากสามารถแก้ไขปัญหานี้ได้ จะช่วยพัฒนาศักยภาพของนักศึกษาได้อย่างรวดเร็วและทำให้พวกขาไปถึงจุดที่ตนเองพอใจได้ง่ายขึ้น เป็นโอกาสที่ดีในการสร้างชื่อให้แก่มหาวิทยาลัย ช่วยให้เห็นแนวทางการพัฒนาตนเองที่ดียิ่งขึ้น และเห็นภาพรวมของสายอาชีพต่าง ๆ

ทางคณะผู้จัดทำจึงมีความสนใจที่จะพัฒนาแพลตฟอร์มสำหรับใช้ช่วยเหลือในการค้นหาสายอาชีพที่เหมาะกับนักศึกษากลุ่มเป้าหมายด้วยปัญญาประดิษฐ์ เพื่อให้กลุ่มเป้าหมายได้ทราบแนวทางของเรซูเมของตนเองในปัจจุบัน พร้อมแนะนำแนวทางการไปเรียนรู้ต่อสำหรับนำไปต่อยอดด้วยตนเองได้

ดังนั้น  หากคณะผู้จัดทำสามารถใช้โครงงานนี้ในการช่วยเหลือนักศึกษาคนอื่น ๆ ในปัจจุบันหรืออนาคต รวมไปถึงคณะผู้จัดทำเองได้ สิ่งนี้จะเป็นประโยชน์แก่นักศึกษาอย่างยิ่งใหญ่ในด้านการศึกษาและการทำงาน เสมือนเป็นตั๋วเบิกทางให้นักศึกษามหาวิทยาลัยเทคโนโลยีพระจอมเกล้าธนบุรี คณะวิศวกรรมศาสตร์ สาขาวิศวกรรมคอมพิวเตอร์ ในอนาคตหลังจากนี้ ซึ่งเป็นแรงจูงใจที่สำคัญยิ่งกับทางคณะผู้จัดทำที่พบปัญหามาด้วยตนเอง และมีอุดมการณ์อยากช่วยเหลือในด้านนี้เช่นกัน


\section{วัตถุประสงค์}

\begin{itemize}
    \item  เพื่อศึกษาและจับจุดสำคัญของข้อมูลภายในเรซูเมของนักศึกษากลุ่มเป้าหมาย
    \item  เพื่อพัฒนาโมเดลปัญญาประดิษฐ์สำหรับวิเคราะห์ข้อมูลเรซูเมของนักศึกษาออกมาเป็นอาชีพที่เหมาะสมกับตัวนักศึกษาได้
    \item  เพื่อพัฒนาเว็บแอปพลิเคชันสำหรับนำมาใช้เป็นตัวกลางในการใช้งานปัญญาประดิษฐ์ และเป็นตัวช่วยด้านการพัฒนาตนเองของกลุ่มเป้าหมายได้
    \item  เพื่อลดปัญหาการค้นหาสิ่งที่เหมาะสมและพัฒนาตนเอง ช่วยเหลือการปรับข้อมูลเรซูเม
\end{itemize}


\section{ขอบเขตของโครงงาน}

ขอบเขตด้านเว็บแอปพลิเคชัน
\begin{itemize}
    \item  เว็บแอปพลิเคชันจะมุ่งเน้นสนับสนุนไปที่เพียง 2 ขนาดหน้าจอ คือ เดสก์ท็อปและมือถือ
\end{itemize}

ขอบเขตด้านปัญญาประดิษฐ์
\begin{itemize}
    \item  ผู้ใช้จะต้องกรอกข้อมูลของเรซูเมด้วยตัวเองเพื่อให้ปัญญาประดิษฐ์วิเคราะห์ผลลัพธ์เป็นวิธีหลัก
    \item  ปัญญาประดิษฐ์จะรองรับภาษาอังกฤษในการวิเคราะห์เท่านั้น
\end{itemize}

ขอบเขตด้านเนื้อหาและกลุ่มเป้าหมาย
\begin{itemize}
    \item  มุ่งเน้นไปที่นักศึกษาวิศวกรรมคอมพิวเตอร์หลักสูตรปกติ และนานาชาติ
    \item  สายอาชีพที่สนับสนุนจะประกอบด้วย 6 สายอาชีพดังนี้
    \begin{itemize}
        \item Developer เช่น frontent developer, backend developer และ full-stack developer
        \item Designer เช่น UX/UI Designer
        \item Data และ AI เช่น Data Engineer, Data Science และ Data Analyst
        \item Security เช่น Data Security และ Cyber Security
        \item Cloud Management เช่น DevOps
        \item Quality Assurance และ Tester เช่น Manual/Automate Tester
    \end{itemize}
    เนื่องจากข้อจำกัดจากข้อมูลที่มีในปัญญาประดิษฐ์ ความนิยมของสายอาชีพนั้น ๆ ในตลาดแรงงาน \cite{springnews} และนัยยะสำคัญที่ได้มาจากกลุ่มเป้าหมาย
\end{itemize}

\section{ประโยชน์ที่คาดว่าจะได้รับ}

ทางคณะผู้จัดทำคาดหวังเป็นอย่างยิ่งว่าจะสามารถนำเว็บแอปพลิเคชันนี้มาใช้งานกับนักศึกษาวิศวกรรมคอมพิวเตอร์ทุกชั้นปีซึ่งรวมถึงคณะผู้จัดทำเองด้วย โดยหวังว่าจะช่วยเหลือให้กลุ่มเป้าหมายนี้ สามารถปรังปรุงเรซูเมของตัวเองให้ดียิ่งขึ้น หรือปรับให้เข้ากับความต้องการของตนเอง เสริมความมั่นใจในการนำไปสมัครงานตามช่องทางที่ตนเองสนใจ รวมไปถึงใช้งานเว็บแอปพลิเคชันเพื่อพัฒนาตนเอง อีกทั้งยังสามารถพัฒนาปัญญาประดิษฐ์ให้มีประสิทธิภาพมากขึ้นจากข้อมูลที่ได้รับระหว่างเปิดให้ใช้งานตามเวลาอีกด้วย ซึ่งเป็นประโยชน์อย่างมากในยุคที่ข้อมูลมีความสำคัญเช่นนี้

\section{ตารางการดำเนินงาน}
\begin{figure}[!h]\centering
    \setlength{\fboxrule}{0.2mm}
    \setlength{\fboxsep}{0.5cm}
    \fbox{\includegraphics[width=15cm]{./figure/figure_ganttchart.png}}
    \caption{ตารางการดำเนินงาน}\label{fig:model4}
\end{figure}


%%%%%%%%%%%%%%%%%%%%%%%%%%%%%%%%%%%%%%%%%%%%%%%%%%%%%%%%%%%%
%%%%%%%%%%%%%%  Literature Review %%%%%%%%%%%%%%%%%%%%%%%%%%
%%%%%%%%%%%%%%%%%%%%%%%%%%%%%%%%%%%%%%%%%%%%%%%%%%%%%%%%%%%%

\chapter{ทฤษฎีความรู้และงานที่เกี่ยวข้อง}

\emph{หัวข้อต่าง ๆ ในแต่ละบทเป็นเพียงตัวอย่างเท่านั้น หัวข้อที่จะใส่ในแต่ละบทขึ้นอยู่กับโปรเจคของนักศึกษาและอาจารย์ที่ปรึกษา}

ตัวอย่างการใส่อ้างอิงที่มา -> \cite{hypersense} ถ้าต้องการใส่แหล่งอ้างอิงมากกว่า 1 ให้ทำดังนี้ -> \cite{hypersense,bworld}
อธิบายทฤษฎี องค์ความรู้หลักที่ใช้ในงาน งานวิจัยที่นำมาใช้ในโครงงาน หรือเปรียบเทียบผลิตภัณฑ์ที่มีอยู่ในท้องตลาด\cite{bworld}
Explain theory, algorithms, protocols, or existing research works and tools related to your work.


\section{ระบบแนะนำสินค้า}

\begin{table}[!h]
    \caption{test table method1}\label{tbl:method1}
    \begin{tabular}{c|c|l|rr} \hline\hline
        Center & Center & left aligned & Right & Right aligned \\ \hline\hline
        Center & Center & left aligned & Right & Right aligned \\ \hline
        Center & Center & left aligned & Right & Right aligned \\
        Center & Center & left aligned & Right & Right aligned \\ \hline
        Center & Center & left aligned & Right & Right aligned \\ \hline\hline
    \end{tabular}
\end{table}


\section{อัลกอริทึมในการประมวลผลข้อความ}
\subsection{อัลกอริทึม I}

% Can define this in the preamble..
You can place the figure and refer to it as รูปที่~\ref{fig:model2}.
The figure and table numbering will be run and updated automatically when you add/remove tables/figures from the document.

\begin{figure}[!h]\centering
    \setlength{\fboxrule}{0.2mm} % can define this in the preamble
    \setlength{\fboxsep}{1cm}
    \fbox{\includegraphics[width=5cm]{./model2.pdf}}
    \caption{The network model}\label{fig:model2}
\end{figure}


\subsection{อัลกอริทึม II}
Add more subsections as you want.
\subsubsection{ขั้นตอนที่ 1}
\subsubsection{ขั้นตอนที่ 2}
Latex Format นี้รองรับหัวข้อย่อยถึงแค่ระดับ 4 นี้เท่านั้น ไม่แนะนำให้แบ่งหัวข้อย่อยไปมากกว่านี้ เช่น 2.2.2.2.1 , 2.2.2.2.2

\section{เครื่องมือที่ใช้ในการพัฒนา}

%%%%%%%%%%%%%%%%%%%%%%%%%%%%%%%%%%%%%%%%%%%%%%%%%%%%%55
%%%%%%%%%%%%%%%%%%%%%%%%%%%%%%%%%%%%%%%%%%%%%%%%%%%%%
%%%%%%%%%%%%%%%%%%%%%%%%%%%%%%%%%%%%%%%%%%%%%%%%%%%%%

\chapter{การออกแบบและวิธีการดำเนินงาน}

\section{การสำรวจความต้องการกับผู้ใช้}
ทางคณะผู้จัดทำได้ทำออกเก็บข้อมูลกับกลุ่มเป้าหมายมาแล้วทั้งหมด 4 ครั้ง โดยแบ่งเป็นการสัมภาษณ์เชิงปริมาณหนึ่งครั้งและเชิงคุณภาพสามครั้ง  โดยมีจุดประสงค์ในแต่ละการสัมภาษณ์ต่างกันเพื่อพิสูจน์ความต้องการของกลุ่มเป้าหมายจนกระทั่งโครงงานของเราได้ปรับตามความต้องการนั้นจนเป็นโครงงานในปัจจุบัน อย่างไรก็ตาม เรายังได้รับข้อมูลที่น่าสนใจเพิ่มเติมโดยสรุป ดังนี้

\subsection{การสัมภาษณ์เชิงปริมาณผ่านแบบสำรวจ}
ทางคณะผู้จัดทำได้ทำแบบสอบถามเพื่อหาอัตราส่วนของพฤติกรรมที่น่าสนใจ โดยได้รับข้อมูลที่สำน่าสนใจดังนี้
\begin{figure}[!h]\centering
    \setlength{\fboxrule}{0.2mm} % can define this in the preamble
    \setlength{\fboxsep}{0.5cm}
    \fbox{\includegraphics[width=10cm]{./figure/figure_poll.png}}
    \caption{ข้อมูลจากแบบสำรวจเชิงปริมาณ}\label{fig:insightPoll}
\end{figure}
จากข้อมูลในแบบสำรวจข้อนี้ ทำให้ทราบว่าปัจจัยที่มีผลต่อการเลือกสายงานในการทำงานมากสำหรับกลุ่มตัวอย่างก็คือเงินเดือน และความสมดุลของการทำงานกับชีวิตส่วนตัว กล่าวคือกลุ่มตัวอย่างให้ความสำคัญกับความจริงมากยิ่งขึ้น ความมั่นคง หรือเงินเดือนที่สามารถทำให้ดำเนินชีวิตได้อย่างราบรื่น รวมถึงการแบ่งเวลา การพักผ่อนที่เหมาะสมกับการทำงาน โดยมองอาจจะไม่ได้ให้ความสำคัญกับความชื่นชอบมากขนาดนั้น

\subsection{การสัมภาษณ์เชิงคุณภาพผ่านการสัมภาษณ์ตัวต่อตัว}
ทางคณะผู้จัดทำได้ออกสัมภาษณ์กับกลุ่มเป้าหมายแบบตัวต่อตัว และในการสัมภาษณ์แต่ละครั้ง จะสัมภาษณ์ที่จำนวนราว 5 คน ตามหลักการ 5 Users design โดยได้รับข้อมูลที่น่าสนใจในแต่ละครั้งมาดังนี้
\newline
\textbf{การสัมภาษณ์เชิงคุณภาพครั้งที่ 1}
\newline
\textbf{จุดประสงค์: เพื่อค้นหารากขอปัญหาที่แท้จริงของกลุ่มเป้าหมาย}
\newline
\textbf{ข้อมูลสำคัญ: }
\begin{itemize}
    \item กลุ่มเป้าหมายค้นพบความต้องการของตนเองมาจากการได้ลงมือทำจริงเป็นหลัก
    \item ส่วนใหญ่แล้วจะได้ลงมือทำจริงตอนโปรเจ็กต์วิชาเรียนหรือฝึกงาน ซึ่งอยู่ชั้นปีที่ 2 เป็นต้นไปแล้ว
    \item อยากรู้ก่อนเป็นอย่างมาก ว่าในการทำงานจริงต้องมีความสามารถอะไรบ้าง ใช้เครื่องมืออะไร
    \item กลุ่มเป้าหมายรู้สึกว่าอาจรู้ตัวช้าเกินไป หากมีโอกาสพัฒนาตนเองได้เร็วกว่านี้จะดีมาก
\end{itemize}


\noindent\textbf{การสัมภาษณ์เชิงคุณภาพครั้งที่ 2}
\newline
\textbf{จุดประสงค์: เพื่อพิสูจน์ความมีคุณภาพของวิธีการแก้ปัญหาที่ออกแบบ}
\newline
\textbf{ข้อมูลสำคัญ: }
\begin{itemize}
    \item กลุ่มเป้าหมายอยากได้ตัวช่วยในการทำให้ตนเองสมัครงานได้ง่ายขึ้น โดยหลังการทดลองถามความเห็น พบว่าสิ่งที่ต้องการเป็นหลักคือ การตรวจสอบและยืนยันได้ ว่าเรซูเม่ของตนเองเหมาะสมกับอาชีพที่ตนเองสนใจขนาดไหนแล้ว
    \item กลุ่มเป้าหมายรู้สึกสนใจในฟีเจอร์ช่วยเหลือการค้นหาแหล่งพัฒนาตนเอง เพราะเคยรู้สึกว่าตนเองอาจเริ่มพัฒนาช้าเกินไป เพราะรู้ใจตัวเองในช่วงที่อาจเรียนอยู่ชั้นปีที่ 2-3 แล้ว
    \item การตัดสินใจลงวิชาเรียนค่อนข้างมีจุดขัดใจ เพราะไม่ค่อยมีรีวิวหรือความเห็นของผู้ที่เคยเรียน ถึงแม้เคยมีแหล่งชุมชนที่รุ่นพี่เคยมอบให้ แต่รีวิวส่วนใหญ่จะมีความเก่าแล้ว ทำให้ใช้อ้างอิงได้ยาก
\end{itemize}


\noindent\textbf{การสัมภาษณ์เชิงคุณภาพครั้งที่ 3}
\newline
\textbf{จุดประสงค์: เพื่อทดลองนำ prototype ของเว็บแอปไปพิสูจน์ความรู้สึกในการใช้งานกับผู้ใช้}
\newline
\textbf{ข้อมูลสำคัญ:}
\begin{itemize}
    \item ผู้ใช้ไม่มีปัญหากับการกรอกข้อมูลเรซูเมเพื่อวิเคราะห์ แต่หากมีระบบที่กรอกข้อมูลให้อัตโนมัติผ่านไฟล์ pdf ก็ถือว่าเป็นเรื่องดี
    \item การมีระบบโหวตวิชาที่อยากให้เปิด อาจไม่ได้เป็นการการันตีว่าจะเปิดได้จริง อาจไม่สำคัญมากนัก
    \item ในอนาคต หากมีระบบที่เป็นตัวช่วยในการสร้างเรซูเมจากข้อมูลที่มีได้ ก็จะเป็นเรื่องที่ดีเช่นกัน
    \item ควรมีการแย่ง tag ประเภทของชุมชนเพื่อความสะดวกในการค้นหา
\end{itemize}

\section{ความสามารถของระบบ}
\subsection{Use Case Diagram}
\begin{figure}[H]\centering
    \setlength{\fboxrule}{0.2mm} % can define this in the preamble
    \setlength{\fboxsep}{0.5cm}
    \fbox{\includegraphics[width=10cm]{./figure/figure_usecase.png}}
    \caption{Use Case Diagram}\label{fig:usecase}
\end{figure}
\subsection{Use Case Narrative}
\subsubsection{Resume Career Prediction}
\begin{table}[H]
    % \centering
    \begin{tabular}{|l|l|} \hline
        Actor                 & Anonymous                                       \\ \hline
        Goal                  & ต้องการทราบถึงอาชีพที่เหมาะสมกับตนเอง                 \\ \hline
        Precondition          & -                                               \\ \hline
        Main success scenario & 1. ผู้ใช้ขอคำแนะนำอาชีพที่เหมาะสมกับตนเอง                \\
                              & 2. ระบบถามข้อมูลของผู้ใช้                            \\
                              & 3. ผู้ใช้กรอกข้อมูล                                  \\
                              & 4. ระบบถามเพื่อยืนยันการกรอกข้อมูล                    \\
                              & 5. ผู้ใช้ยืนยันการกรอกข้อมูล                           \\
                              & 6. ระบบแสดงอาชีพที่เหมาะสมกับผู้ใช้                    \\ \hline
        Extensions (a)        & 5a. ผู้ใช้กรอกข้อมูลไม่ครบ                            \\
                              & 6a. ระบบขึ้นเตือนว่าผู้ใช้ยังกรอกข้อมูลไม่ครบ              \\
                              & 7a. กลับไปที่ขั้นตอนที่ 3                              \\ \hline
        Postcodition          & ระบบแนะนำให้ผู้ใช้ไปวิเคราะห์ข้อมูลเชิงลึกของอาชีพที่ได้แนะนำไป \\ \hline
    \end{tabular}
\end{table}

\subsubsection{See career path node tree}
\begin{table}[H]
    % \centering
    \begin{tabular}{|l|l|} \hline
        Actor                 & Anonymous                                              \\ \hline
        Goal                  & ต้องการดูแผนผังสายอาชีพ                                    \\ \hline
        Precondition          & ผู้ใช้กำลังดูแผนผังรวมของสายอาชีพ                              \\ \hline
        Main success scenario & 1. ใช้เลือกอาชีพที่ต้องการจะดู                                \\
                              & 2. ระบบแสดงรายละเอียดย่อยเป็นรูปแบบแผนผังของสายอาชีพที่ผู้ใช้เลือก \\ \hline
        Extensions (a)        & 1a. ผู้ใช้ยังไม่ได้เลือกอาชีพที่ต้องการจะดู                        \\
                              & 2a. ระบบขึ้นเตือนว่าผู้ใช้ยังไม่ได้เลือกอาชีพที่ต้องการดู              \\
                              & 3a. กลับไปที่ขั้นตอนที่ 1                                     \\ \hline
        Postcodition          & -                                                      \\ \hline
    \end{tabular}
\end{table}

\subsubsection{View Community}
\begin{table}[H]
    % \centering
    \begin{tabular}{|l|l|} \hline
        Actor                 & Anonymous                              \\ \hline
        Goal                  & ต้องการดูรายการของกระทู้ต่าง ๆ ในชุมชน       \\ \hline
        Precondition          & ผู้ใช้กำลังดูประเภทของกระทู้                   \\ \hline
        Main success scenario & 1. ผู้ใช้เลือกประเภทของกระทู้                \\
                              & 2. ระบบแสดงรายการของกระทู้ประเภทที่ผู้ใช้เลือก \\ \hline
        Extensions (a)        & -                                      \\ \hline
        Postcodition          & -                                      \\ \hline
    \end{tabular}
\end{table}

\subsubsection{View Topic/Forum}
\begin{table}[H]
    % \centering
    \begin{tabular}{|l|l|} \hline
        Actor                 & Anonymous                                        \\ \hline
        Goal                  & ต้องการดูรายละเอียดของกระทู้                          \\ \hline
        Precondition          & ผู้ใช้กำลังดูรายการของกระทู้ที่อยู่ในชุมชน (กระทู้ประเภทไหนก็ได้) \\ \hline
        Main success scenario & 1. ผู้ใช้เลือกกระทู้ที่ต้องการดูรายละเอียด                  \\
                              & 2. ระบบแสดงรายละเอียดของกระทู้ผู้ใช้เลือก               \\ \hline
        Extensions (a)        & -                                                \\ \hline
        Postcodition          & -                                                \\ \hline
    \end{tabular}
\end{table}

\subsubsection{Create Topic/Forum}
\begin{table}[H]
    % \centering
    \begin{tabular}{|l|l|} \hline
        Actor                 & Logged-in User                                   \\ \hline
        Goal                  & ต้องการสร้างกระทู้                                   \\ \hline
        Precondition          & ผู้ใช้กำลังดูรายการของกระทู้ที่อยู่ในชุมชน (กระทู้ประเภทไหนก็ได้) \\ \hline
        Main success scenario & 1. ผู้ใช้เลือกกระทู้ที่ต้องการดูรายละเอียด                  \\
                              & 2. ระบบถามรายละเอียดภายในกระทู้                     \\
                              & 3. ผู้ใช้กรอกรายอะเอียดภายในกระทู้                     \\
                              & 4. ระบบถามเพื่อยืนยันการลงกระทู้                       \\
                              & 5. ผู้ใช้ยืนยันการลงกระทู้                              \\
                              & 6. ระบบสร้างและแสดงกระทู้ของผู้ใช้                     \\ \hline
        Extensions (a)        & 5a. ผู้ใช้กรอกข้อมูลไม่ครบ                             \\
                              & 6a. ระบบขึ้นเตือนว่าผู้ใช้ยังกรอกข้อมูลไม่ครบ               \\
                              & 7a. กลับไปที่ขั้นตอนที่ 3                               \\ \hline
        Postcodition          & กระทู้ของผู้ใช้แสดงอยู่ในชุมชน                           \\ \hline
    \end{tabular}
\end{table}

\subsubsection{Reply Topic/Forum}
\begin{table}[H]
    % \centering
    \begin{tabular}{|l|l|} \hline
        Actor                 & Logged-in User                                   \\ \hline
        Goal                  & ต้องการแสดงความคิดเห็นในกระทู้                        \\ \hline
        Precondition          & ผู้ใช้กำลังดูรายการของกระทู้ที่อยู่ในชุมชน (กระทู้ประเภทไหนก็ได้) \\ \hline
        Main success scenario & 1. ผู้ใช้เลือกส่วนที่ต้องการแสดงความคิดเห็น                \\
                              & 2. ระบบถามถึงรายละเอียดความคิดเห็น                   \\
                              & 3. ผู้ใช้กรอกความคิดเห็น                              \\
                              & 4. ระบบถามเพื่อยืนยันการแสดงความคิดเห็น                \\
                              & 5. ผู้ใช้ยืนยันการแสดงความคิดเห็น                       \\
                              & 6. ระบบสร้างและแสดงความคิดเห็นของผู้ใช้ในกระทู้          \\ \hline
        Extensions (a)        & 4a. ผู้ใช้ยังไม่ได้กรอกความคิดเห็น                       \\
                              & 5a. ระบบขึ้นเตือนว่าผู้ใช้ยังไม่ได้กรอกความคิดเห็น           \\
                              & 6a. กลับไปที่ขั้นตอนที่ 3                               \\ \hline
        Postcodition          & ความคิดเห็นของผู้ใช้แสดงอยู่ในกระทู้                      \\ \hline
    \end{tabular}
\end{table}

\subsubsection{Upload Resume}
\begin{table}[H]
    % \centering
    \begin{tabular}{|l|l|} \hline
        Actor                 & Logged-in User                                  \\ \hline
        Goal                  & ต้องการอัพโหลดเรซูเม่                               \\ \hline
        Precondition          & ผู้ใช้กำลังอยู่ดูข้อมูลส่วนตัวของตนเอง                      \\ \hline
        Main success scenario & 1. ผู้ใช้เลือกเรซูเม่ที่ต้องการอัพโหลด                    \\
                              & 2. ระบบถามเพื่อยืนยันการอัพโหลดเรซูเม่                 \\
                              & 3. ผู้ใช้ยืนยันการอัพโหลดเรซูเม่                        \\
                              & 4. ระบบอัพโหลดและแจ้งเตือนว่าอัพโหลดเรซูเม่สำเร็จ        \\ \hline
        Extensions (a)        & 3a. ผู้ใช้เลือกเรซูเม่ที่ต้องการอัพโหลด                   \\
                              & 4a. ระบบขึ้นเตือนว่าผู้ใช้ยังไม่ได้เลือกเรซูเม่ที่ต้องการอัพโหลด \\
                              & 5a. กลับไปที่ขั้นตอนที่ 1                              \\ \hline
        Postcodition          & เรซูเม่แสดงอยู่ในข้อมูลส่วนตัวของผู้ใช้                    \\ \hline
    \end{tabular}
\end{table}

\subsubsection{Analyzed Resume}
\begin{table}[H]
    % \centering
    \begin{tabular}{|l|l|} \hline
        Actor                 & Logged-in User                  \\ \hline
        Goal                  & ต้องการดูข้อมูลเชิงลึกของอาชีพ         \\ \hline
        Precondition          & ผู้ใช้ต้องมีอาชีพที่ระบบแนะนำให้          \\ \hline
        Main success scenario & 1. ผู้ใช้เลือกอาชีพที่ต้องการดูข้อมูลเชิงลึก \\
                              & 2. ระบบแสดงข้อมูลเชิงลึกของอาชีพ     \\ \hline
        Extensions (a)        & 1a. ผู้ใช้ต้องวิเคราะห์อาชีพใหม่        \\
                              & 2a. ระบบถามถึงข้อมูลของผู้ใช้         \\
                              & 3a. ผู้ใช้กรอกข้อมูลใหม่              \\
                              & 4a. ผู้ใช้ยืนยันการกรอกข้อมูล          \\
                              & 5a. ระบบแสดงอาชีพที่เหมาะสมกับผู้ใช้   \\
                              & 6a. กลับไปที่ขั้นตอนที่ 1              \\ \hline
        Postcodition          & -                               \\ \hline
    \end{tabular}
\end{table}

\section{สถาปัตยกรรมของระบบ}
\begin{figure}[H]\centering
    \setlength{\fboxrule}{0.2mm} % can define this in the preamble
    \setlength{\fboxsep}{0.5cm}
    \fbox{\includegraphics[width=10cm]{./figure/figure_system_architecture.png}}
    \caption{System Architecture}\label{fig:system_architecture}
\end{figure}
จากแผนผังสถาปัตยกรรมระบบ เราได้ใช้บริการ Google Cloud เป็นหลักเนื่องจากสะดวกต่อการ scaling ในอนาคต สามารถจัดการและเปลี่ยนแปลงเวอร์ชันได้รวดเร็วระหว่างการพัฒนา โดยสถาปัตยกรรมแต่ละส่วนมีหน้าที่หลักดังนี้

\subsection{ส่วนผู้ใช้ (Client-Side)}
รับผิดชอบในการติดต่อปฏิสัมพันธ์กับผู้ใช้ แสดง user interface ผ่านทาง web browser และเชื่อมต่อกับ server เพื่อขอข้อมูลและใช้บริการต่าง ๆ

\subsection{ส่วนเซิร์ฟเวอร์ (Server-Side)}
รับผิดชอบในการจัดการตรรกะและระบบคำนวณต่าง ๆ ทุกรูปแบบและส่งกลับไปให้ผู้ใช้ โดยทางคณะู้จัดทำได้แบ่งระบบ server เป็นสองบริการหลัก ประกอบด้วย
\newline
1. บริการหลัก (Main services)
\newline
บริการที่จะติดต่อกับผู้ใช้โดยตรงเพียงหนึ่งเดียว มีหน้าที่ในการจัดการทุกคำขอจากผู้ใช้ผ่านทาง API (Application Programming Interfaces) เช่น ดึงข้อมูลจากฐานข้อมูล คำนวณค่าต่าง ๆ และติดต่อกับบริการอื่น ๆ
\newline
2. บริการโมเดล (Model services)
\newline
บริการสำหรับใช้คำนวณเกี่ยวกับปัญญาประดิษฐ์ของทางคณะผู้จัดทำเท่านั้น เนื่องจากกินทรัพยากรสูง การแยกบริการออกมาจากบริการหลักจึงสามารถดูแลและจัดการได้ง่ายกว่า โดยจะรับค่ามาจากส่วนบริการหลักและส่งค่ากลับไปให้

\newpage

\section{Site Map}
\begin{figure}[h]\centering
    \setlength{\fboxrule}{0.2mm} % can define this in the preamble
    \setlength{\fboxsep}{0.5cm}
    \fbox{\includegraphics[width=10cm]{./figure/figure_siteMap.png}}
    \caption{SiteMap}\label{fig:siteMap}
\end{figure}

\section{Navigation Map}
\begin{figure}[h]\centering
    \setlength{\fboxrule}{0.2mm} % can define this in the preamble
    \setlength{\fboxsep}{0.5cm}
    \fbox{\includegraphics[width=10cm]{./figure/figure_navMap.png}}
    \caption{Navigation Map}\label{fig:navMap}
\end{figure}

\section{Wireframe}
\begin{table}[H]
\begin{tabular}{|l|}
\hline
Resume Checker                                                                                                                                                                                                                 \\ \hline
\begin{minipage}{\linewidth}
  \begin{itemize}
    \item หน้าหลัก
  \end{itemize}
\end{minipage}                                                                                                                                                                                                       \\ \hline
 \begin{minipage}{\linewidth}
    \centering
    \vspace{1em} % Adjust the space as needed
  \fbox{\includegraphics[width=10cm]{./figure/figure_wireframe_table1_mainpage1.png}}
  \caption{\centering หน้าหลักสำหรับกดเริ่มวิเคราะห์ Resume}\label{fig:wireframe1_1}
\end{minipage} \\                          
  \begin{minipage}{\linewidth}
    \centering
    \vspace{1em} % Adjust the space as needed
  \fbox{\includegraphics[width=10cm]{./figure/figure_wireframe_table1_mainpage2.png}}
  \caption{\centering หน้ากรอกข้อมูลของเพื่อนำไปวิเคราะห์}\label{wireframe1_2}
\end{minipage} \\                                                                                                                                                                                                                            \\ \hline
\end{tabular}
\end{table}

\begin{table}[H]
\begin{tabular}{|l|}
\hline
\begin{minipage}{\linewidth}
  \begin{itemize}
    \item ส่วนประกอบแบบโต้ตอบ
  \end{itemize}
\end{minipage}                                                                                                                                                                                                               \\ \hline
  \begin{minipage}{\linewidth}
    \centering
    \vspace{1em} % Adjust the space as needed
  \fbox{\includegraphics[width=6cm]{./figure/figure_wireframe_table1_component1.png}}
  \caption{\centering ช่องสำหรับกรอกข้อมูลของเพื่อนำไปวิเคราะห์}\label{wireframe1_3}
\end{minipage} \\                                                                                                                                                                                                                                               \\ \hline
\begin{minipage}{\linewidth}
  \begin{itemize}
    \item คำอธิบาย
  \end{itemize}
\end{minipage}                                                                                                                                                                                                                       \\ \hline
\begin{minipage}{\linewidth}
  \raggedright
  Resume Checker เป็นฟีเจอร์ที่ให้ผู้ใช้กรอกข้อมูลของเพื่อนำไปวิเคราะห์อาชีพที่เหมาะสม โดยผู้ใช้สามารถการข้อมูลเกี่ยวกับประสบการณ์ทำงาน, การศึกษา, ทักษะ, และคุณสมบัติอื่น ๆ ที่สำคัญ
\end{minipage}
 \\ \hline
\end{tabular}
\end{table}


\begin{table}[H]
\begin{tabular}{|l|}
\hline
Match Career                                                                                                                                                                                                                                                                                                    \\ \hline
\begin{minipage}{\linewidth}
  \begin{itemize}
    \item หน้าหลัก
  \end{itemize}
\end{minipage}                                                                                                                                                                                                                                                                                                 \\ \hline
   \begin{minipage}{\linewidth}
    \centering
    \vspace{1em} % Adjust the space as needed
  \fbox{\includegraphics[width=10cm]{./figure/figure_wireframe_table2_mainpage1.png}}
  \caption{\centering ช่องสำหรับกรอกข้อมูลของเพื่อนำไปวิเคราะห์}\label{fig:wireframe2_1}
\end{minipage} \\                                                                                                                                                                                                                                                                                                                                   \\ \hline
\begin{minipage}{\linewidth}
  \begin{itemize}
    \item คำอธิบาย
  \end{itemize}
\end{minipage}                                                                                                                                                                                                                                                                                                                                                                                                                                                                                                                                 \\ \hline
\begin{minipage}{\linewidth}
  \raggedright
Match Career เป็นฟีเจอร์สำหรับแสดงอาชีพที่เหมาะสมกับข้อมูลที่ผู้ใช้กรอกมาทำให้ผู้ใช้สามารถรู้ถึงอาชีพที่เหมาะสมกับตนเองและข้อมูลคร่าว ๆ ของอาชีพนั้น โดยผู้ใช้สามารถกลับไปแก้ไขข้อมูลที่กรอกได้เพื่อนำมาวิเคราะห์อาชีพใหม่ รวมถึงสามารถกดวิเคราะห์ข้อมูลเชิงลึกต่อไปได้แต่ผู้ใช้ต้องสมัครและล็อคอินเพื่อเก็บข้อมูลก่อน
\end{minipage}
\\ \hline
\end{tabular}
\end{table}

\begin{table}[H]
\begin{tabular}{|l|}
\hline
Evaluate Result                                                                                                                                                                                                                                                                                                   \\ \hline
\begin{minipage}{\linewidth}
  \begin{itemize}
    \item หน้าหลัก
  \end{itemize}
\end{minipage}                                                                                                                                                                                                                                                                                                                      \\ \hline
  \begin{minipage}{\linewidth}
    \centering
    \vspace{1em} % Adjust the space as needed
  \fbox{\includegraphics[width=10cm]{./figure/figure_wireframe_table3_mainpage1.png}}
  \caption{\centering ช่องสำหรับกรอกข้อมูลของเพื่อนำไปวิเคราะห์}\label{fig:wireframe3_1}
\end{minipage} \\                                                                                                                                                                                                                                                                                                                                    \\ \hline
\begin{minipage}{\linewidth}
  \begin{itemize}
    \item คำอธิบาย
  \end{itemize}
\end{minipage}                                                                                                                                                                                                                                                                                                                                                                                                                                                                                                                                 \\ \hline
\begin{minipage}{\linewidth}
  \raggedright
Evaluate Result เป็นฟีเจอร์สำหรับแสดงข้อมูลเชิงลึกของอาชีพที่วิเคราะห์ได้ในตอนแรก โดยจะแสดงข้อมูลที่มีโยชน์เพื่อให้ผู้ใช้สามารถนำไปพัฒนาทักษะของตนเองและเรูเม่ต่อไปได้
\end{minipage}
\\ \hline
\end{tabular}
\end{table}

\begin{table}[H]
\begin{tabular}{|l|}
\hline
Career Node                                                                                                                                                                                                                                                                                              \\ \hline
\begin{minipage}{\linewidth}
  \begin{itemize}
    \item หน้าหลัก
  \end{itemize}
\end{minipage}                                                                                                                                                                                                                                                                                                                 \\ \hline
 \begin{minipage}{\linewidth}
    \centering
    \vspace{1em} % Adjust the space as needed
  \fbox{\includegraphics[width=10cm]{./figure/figure_wireframe_table4_mainpage1.png}}
  \caption{\centering ช่องสำหรับกรอกข้อมูลของเพื่อนำไปวิเคราะห์}\label{fig:wireframe4_1}
\end{minipage} \\                                                                                                                                                                                                                                                                                                                                                    \\ \hline
\begin{minipage}{\linewidth}
  \begin{itemize}
    \item คำอธิบาย
  \end{itemize}
\end{minipage}                                                                                                                                                                                                                                                                                                                                                                                                                                                                                                                            \\ \hline
\begin{minipage}{\linewidth}
  \raggedright
Career Node เป็นฟีเจอร์สำหรับแสดงทักษะที่ต้องใช้ในอาชีพต่าง ๆ โดยจะแสดงออกมาในรูปแบบโหนด
\end{minipage}
\\ \hline
\end{tabular}
\end{table}

\begin{table}[H]
\begin{tabular}{|l|}
\hline
Community                                                                                                                                                                                                                                                                                          \\ \hline
\begin{minipage}{\linewidth}
  \begin{itemize}
    \item หน้าหลัก
  \end{itemize}
\end{minipage}                                                                                                                                                                                                                                                                                                                        \\ \hline
 \begin{minipage}{\linewidth}
    \centering
    \vspace{1em} % Adjust the space as needed
  \fbox{\includegraphics[width=10cm]{./figure/figure_wireframe_table5_mainpage1.png}}
  \caption{\centering ช่องสำหรับกรอกข้อมูลของเพื่อนำไปวิเคราะห์}\label{fig:wireframe5_1}
\end{minipage} \\       
 \begin{minipage}{\linewidth}
    \centering
    \vspace{1em} % Adjust the space as needed
  \fbox{\includegraphics[width=10cm]{./figure/figure_wireframe_table5_mainpage2.png}}
  \caption{\centering ช่องสำหรับกรอกข้อมูลของเพื่อนำไปวิเคราะห์}\label{fig:wireframe5_2}
\end{minipage} \\                           
\\ \hline
\end{tabular}
\end{table}

\begin{table}[H]
\begin{tabular}{|l|}
\hline
\begin{minipage}{\linewidth}
    \centering
    \vspace{1em} % Adjust the space as needed
  \fbox{\includegraphics[width=10cm]{./figure/figure_wireframe_table5_mainpage3.png}}
  \caption{\centering ช่องสำหรับกรอกข้อมูลของเพื่อนำไปวิเคราะห์}\label{fig:wireframe5_3}
\end{minipage} \\                                                                                                                                                                                                                                                                                                       \\ \hline
\begin{minipage}{\linewidth}
  \begin{itemize}
    \item คำอธิบาย
  \end{itemize}
\end{minipage}                                                                                                                                                                                                                                                                                            \\ \hline
\begin{minipage}{\linewidth}
  \raggedright
Community เป็นฟีเจอร์สำหรับให้ผู้ใช้สามารถเข้ามาแลกเปลี่ยนข้อมูลหรือความคิดเห็นได้ผ่านการตั้งกระทู้และแสดงความคิดเห็น
\end{minipage}
\\ \hline
\end{tabular}
\end{table}




\section{สถาปัตยกรรมของระบบ}
\begin{figure}[H]\centering
    \setlength{\fboxrule}{0.2mm} % can define this in the preamble
    \setlength{\fboxsep}{0.5cm}
    \fbox{\includegraphics[width=10cm]{./figure/figure_system_architecture.png}}
    \caption{System Architecture}\label{fig:model4}
\end{figure}


% \emph{หัวข้อต่าง ๆ ในแต่ละบทเป็นเพียงตัวอย่างเท่านั้น หัวข้อที่จะใส่ในแต่ละบทขึ้นอยู่กับโปรเจคของนักศึกษาและอาจารย์ที่ปรึกษา}


% ตัวอย่างการใส่อ้างอิงที่มา -> \cite{hypersense} ถ้าต้องการใส่แหล่งอ้างอิงมากกว่า 1 ให้ทำดังนี้ -> \cite{hypersense,bworld}
% Explain the design (how you plan to implement your work) of your project. Adjust the section titles below to suit the types of your work. Detailed physical design like circuits and source codes should be placed in the appendix.

% \section{ข้อกำหนดและความต้องการของระบบ}

% \section{สถาปัตยกรรมระบบ}

% \begin{table}[!h]
%     % \centering
%     \caption{test table x1}\label{tbl:symbols}
%     \begin{tabular}{@{}p{0.07\textwidth}|p{0.7\textwidth}p{0.1\textwidth}}\hline
%         \multicolumn{2}{l}{\textbf{SYMBOL}} & \textbf{UNIT}                         \\ \hline
%         $\alpha$                            & Test variable\hfill     & m$^2$       \\
%         $\lambda$                           & Interarrival rate\hfill & jobs/second \\
%         $\mu$                               & Service rate\hfill      & jobs/second \\ \hline
%     \end{tabular}
%     %\begin{tabular}{c|c} \hline
%     % $\alpha$ & $\beta$ \\ \hline
%     % $\delta$ & $\mu$ \\ \hline
%     %\end{tabular}
% \end{table}



% \section{Hardware Module 1}
% \subsection{Component 1}
% \subsection{Logical Circuit Diagram}

% \section{Hardware Module 2}
% \subsection{Component 1}
% \subsection{Component 2}

% \section{Path Finding Algorithm}

% \section{Database Design}

% \section{UML Design}

% \section{GUI Design}

% \section{การออกแบบการทดลอง}
% \subsection{ตัวชี้วัดและปัจจัยที่ศึกษา}
% \subsection{รูปแบบการเก็บข้อมูล}

%%%%%%%%%%%%%%%%%%%%%%%%%%%%%%%%%%%%%%%%%%%%%%%%%%%%%%%%%%%%%%
%%%%%%%%%%%%%%%%%%%% Experiments %%%%%%%%%%%%%%%%%%%%%%%%%%%%%
%%%%%%%%%%%%%%%%%%%%%%%%%%%%%%%%%%%%%%%%%%%%%%%%%%%%%%%%%%%%%%%

\chapter{ผลการดำเนินงาน}

\section{Software Testing}
\subsection{Test Requirements}
\begin{enumerate}
    \item \textbf{Career Prediction (ทำนายสายอาชีพ)}
          \begin{itemize}
              \item ผู้ใช้งานสามารถทำนายสายอาชีพของผู้ใช้งานจากเรซูเมได้ โดยพิจารณาจากประวัติการศึกษา, ทักษะความรู้, ทักษะอารมณ์ รวมไปถึงประสบการณ์ทำงานหรือหน้าที่ที่เคยรับผิดชอบโปรเจ็คภายในมหาวิทยาลัย
              \item ผู้ใช้งานจำเป็นต้องยินยอมในการจัดเก็บข้อมูลก่อน ถึงจะสามารถทำนายได้
              \item ผู้ใช้งานสามารถแก้ไขรายละเอียดในเรซูเมเพื่อทำนายสายอาชีพใหม่อีกครั้งได้
              \item ผู้ใช้งานสามารถบันทึกประวัติการทำนายได้
              \item ผู้ใช้งานสามารถดูข้อมูลเชิงลึกของสายอาชีพที่ตนเองถูกทำนายได้
              \item ผู้ใช้งานสามารถลบประวัติการทำนายได้
          \end{itemize}
    \item \textbf{Career Insight (ข้อมูลเชิงลึกของสายอาชีพ)}
          \begin{itemize}
              \item ผู้ใช้งานสามารถอ่านรายละเอียดเชิงลึกของแต่ละสายอาชีพได้ ซึ่งจะประกอบไปด้วยอาชีพที่เกี่ยวข้องภายในสายนั้น ๆ
              \item ผู้ใช้งานสามารถสับเปลี่ยนดูรายละเอียดของสายอาชีพที่เคยถูกทำนายไปแล้วได้
              \item ผู้ใช้งานสามารถดูทักษะที่เหมาะสม ทักษะที่ควรเรียนรู้ และทักษะที่เกี่ยวข้องอื่นได้
              \item ผู้ใช้งานสามารถเปรียบเทียบสายอาชีพที่ทำนาย กับสายอาชีพที่ต้องการได้
              \item ผู้ใช้งานสามารถดูสายใยอาชีพโดยโฟกัสจากอาชีพที่ถูกทำนายได้
          \end{itemize}
    \item \textbf{Career Exploration (สายใยอาชีพ)}
          \begin{itemize}
              \item ผู้ใช้งานสามารถรับชมสายใยอาชีพทั้งหมดภายในเว็บไซต์ โดยจะประกอบไปด้วย ชื่อสายอาชีพ, อาชีพที่เกี่ยวข้อง, ทักษะที่เกี่ยวข้อง
              \item ผู้ใช้งานสามารถเลือกอาชีพที่สนใจ เพื่อดูทักษะที่จำเป็นในอาชีพนั้น ๆ ได้
          \end{itemize}
\end{enumerate}
\label{sec:test-subsection}
\subsection{Test Environment}
\begin{enumerate}
    \item \textbf{Date} : April 2024
    \item \textbf{Opertaion System} : Windows 11 (Version 23H2)
    \item \textbf{Browser} : Microsoft Edge (Version 122.0.2365.92)
    \item \textbf{Screen Resolution} : 1920 pixels x 1080 pixels (Scale 125\%)
    \item \textbf{Web-application version} : \href{https://compath-cpe.web.app/}{Compath-MVP (Production Server)}
\end{enumerate}
\subsection{Test Scenario}
% \begin{table}[h!]
\begin{enumerate}
    \item Career Prediction
          \begin{longtable}{|>{\raggedright\arraybackslash}p{0.1\linewidth}|>{\raggedright\arraybackslash}p{0.15\linewidth}|>{\raggedright\arraybackslash}p{0.17\linewidth}|>{\raggedright\arraybackslash}p{0.17\linewidth}|>{\centering}p{0.1\linewidth}|>{\raggedright\arraybackslash}p{0.1\linewidth}|} \hline
              หัวข้อ                   & กรณีทดสอบ                              & ผลที่คาดว่าจะได้รับ                              & ผลที่ได้รับ                                     & ผลลัพธ์ & หมายเหตุ \\ \hline
              \endhead
              ระบบทำนายสายอาชีพ        & กรอกรายละเอียดครบตามช่องที่กำหนด           & สามารถทำนายสายอาชีพได้                         & สามารถทำนายสายอาชีพได้                         & ผ่าน   &         \\ \cline{2-6}& กรอกรายละเอียดครบตามช่องที่กำหนด           & สามารถทำนายสายอาชีพได้                         & สามารถทำนายสายอาชีพได้                         & ผ่าน   &         \\ \cline{2-6}
                                     & กรอกรายละเอียดไม่ครบตามช่องที่กำหนด         & ไม่สามารถทำนายสายอาชีพได้                       & ไม่สามารถทำนายสายอาชีพได้                       & ผ่าน   &         \\ \cline{2-6}
                                     & ยินยอมการจัดเก็บข้อมูล                     & สามารถทำนายสายอาชีพได้                         & สามารถทำนายสายอาชีพได้                         & ผ่าน   &         \\ \cline{2-6}
                                     & ไม่ยินยอมการจัดเก็บข้อมูล                   & แจ้งเตือนให้ยินยอมก่อน จึงจะสามารถทำนายได้          & แจ้งเตือนให้ยินยอมก่อน จึงจะสามารถทำนายได้          & ผ่าน   &         \\ \hline
              ระบบจัดการผลการทำนายอาชีพ & บันทึกผลการทำนายสายอาชีพ                  & สามารถบันทึกผลการทำนายสายอาชีพได้                & สามารถบันทึกผลการทำนายสายอาชีพได้                & ผ่าน   &         \\ \cline{2-6}
                                     & แก้ไขผลการทำนายสายอาชีพ                  & สามารถแก้ไขผลการทำนายสายอาชีพได้                & สามารถแก้ไขผลการทำนายสายอาชีพได้                & ผ่าน   &         \\ \cline{2-6}
                                     & ยกเลิกผลการทำนายสายอาชีพ                 & สามารถยกเลิกผลการทำนายสายอาชีพได้               & สามารถยกเลิกผลการทำนายสายอาชีพได้               & ผ่าน   &         \\ \cline{2-6}
                                     & อ่านรายละเอียดเชิงลึกของสายอาชีพที่ทำนายได้    & สามารถอ่านรายละเอียดเชิงลึกของสายอาชีพที่ทำนายได้    & สามารถอ่านรายละเอียดเชิงลึกของสายอาชีพที่ทำนายได้    & ผ่าน   &         \\ \hline
              ระบบประวัติการทำนาย        & รายละเอียดโดยย่อของประวัติการทำนาย         & แสดงรายละเอียดโดยย่อของประวัติการทำนายได้ถูกต้อง    & แสดงรายละเอียดโดยย่อของประวัติการทำนายได้ถูกต้อง    & ผ่าน   &         \\ \cline{2-6}
                                     & อ่านรายละเอียดเชิงลึกของสายอาชีพที่เคยทำนายไว้ & สามารถอ่านรายละเอียดเชิงลึกของสายอาชีพที่เคยทำนายไว้ & สามารถอ่านรายละเอียดเชิงลึกของสายอาชีพที่เคยทำนายไว้ & ผ่าน   &         \\ \cline{2-6}
                                     & ลบประวัติที่เคยทำนายไว้                     & สามารถลบประวัติที่เคยทำนายไว้                     & สามารถลบประวัติที่เคยทำนายไว้                     & ผ่าน   &         \\ \hline
              \caption{ตารางข้อมูลการทดสอบระบบ Career Prediction}
              \label{tbl:test-scenario-cprediction}
          \end{longtable}
    \item Career Insight
          \begin{longtable}{|>{\raggedright\arraybackslash}p{0.1\linewidth}|>{\raggedright\arraybackslash}p{0.15\linewidth}|>{\raggedright\arraybackslash}p{0.17\linewidth}|>{\raggedright\arraybackslash}p{0.17\linewidth}|>{\centering}p{0.1\linewidth}|>{\raggedright\arraybackslash}p{0.1\linewidth}|} \hline
              หัวข้อ                     & กรณีทดสอบ                                       & ผลที่คาดว่าจะได้รับ                                                                    & ผลที่ได้รับ                                                                            & ผลลัพธ์ & หมายเหตุ                      \\ \hline
              \endhead
              รายละเอียดเชิงลึกของสายอาชีพ & รายละเอียดที่แสดงตรงตามกับสายอาชีพที่ถูกทำนายได้         & รายละเอียดตรงกับสายอาชีพที่ทำนายได้                                                      & รายละเอียดตรงกับสายอาชีพที่ทำนายได้                                                       & ผ่าน   &                              \\ \cline{2-6}
                                       & รายละเอียดที่แสดงมีความถูกต้อง ตรงกับในฐานข้อมูล        & รายละเอียดตรงกับในฐานข้อมูล                                                           & รายละเอียดตรงกับในฐานข้อมูล                                                            & ผ่าน   &                              \\ \cline{2-6}
                                       & สลับอาชีพที่เกี่ยวข้องกับสายอาชีพที่ทำนายได้                & รายละเอียดของอาชีพถูกสับเปลี่ยนตามที่เลือก                                                 & รายละเอียดของอาชีพถูกสับเปลี่ยนตามที่เลือก                                                  & ผ่าน   &                              \\ \cline{2-6}
                                       & รายละเอียดทักษะที่เหมาะสม, ทักษะที่ควรเรียนรู้, ทักษะอื่น ๆ & แสดงรายละเอียดทักษะที่เหมาะสม, ทักษะที่ควรเรียนรู้, ทักษะอื่น ๆ ได้ถูกต้องตามรายละเอียดที่กรอก       & แสดงรายละเอียดทักษะที่เหมาะสม, ทักษะที่ควรเรียนรู้, ทักษะอื่น ๆ ไม่ได้ถูกต้องตามรายละเอียดที่กรอกทั้งหมด & ไม่ผ่าน & ไม่สามารถจับคู่กับคำที่พ้องความหมายได้ \\ \cline{2-6}
                                       & เปรียบเทียบสายอาชีพที่เหมาะสม กับสายอาชีพที่ต้องการ      & สามารถเปรียบเทียบรายละเอียดทักษะที่เหมาะสม, ทักษะที่ควรเรียนรู้, ทักษะอื่น ๆ กับสายงานอื่นที่ต้องการได้ & สามารถเปรียบเทียบรายละเอียดทักษะที่เหมาะสม, ทักษะที่ควรเรียนรู้, ทักษะอื่น ๆ กับสายงานอื่นที่ต้องการได้  & ผ่าน   &                              \\ \hline
              เลือกประวัติการทำนาย             & เลือกประวัติการทำนายที่เคยทำนายไปแล้ว                  & สามารถเลือกประวัติการทำนายที่เคยทำนายไปแล้วได้                                             & สามารถเลือกประวัติการทำนายที่เคยทำนายไปแล้วได้                                              & ผ่าน   &                              \\ \hline
              อาชีพและทักษะที่เกี่ยวข้อง      & แสดงภาพรวมของอาชีพ และทักษะที่เกี่ยวข้องอื่น ๆ          & สามารถดูภาพรวมของอาชีพ และทักษะที่เกี่ยวข้องอื่น ๆ ได้ ในหน้าของสายใยอาชีพได้                   & สามารถดูภาพรวมของอาชีพ และทักษะที่เกี่ยวข้องอื่น ๆ ได้ ในหน้าของสายใยอาชีพได้                    & ผ่าน   &                              \\ \hline
              \caption{ตารางข้อมูลการทดสอบระบบ Career Insight}
              \label{tbl:test-scenario-cinsight}
          \end{longtable}
    \item Career Exploration
          \begin{longtable}{|>{\raggedright\arraybackslash}p{0.1\linewidth}|>{\raggedright\arraybackslash}p{0.15\linewidth}|>{\raggedright\arraybackslash}p{0.17\linewidth}|>{\raggedright\arraybackslash}p{0.17\linewidth}|>{\centering}p{0.1\linewidth}|>{\raggedright\arraybackslash}p{0.1\linewidth}|} \hline
              หัวข้อ           & กรณีทดสอบ                                          & ผลที่คาดว่าจะได้รับ                                           & ผลที่ได้รับ                                                  & ผลลัพธ์ & หมายเหตุ    \\ \hline
              \endhead
              แผนผังสายใยอาชีพ & แสดงรายละเอียดของอาชีพ และทักษะที่เกี่ยวข้องในแต่ละสายอาชีพ & แสดงรายละเอียดของอาชีพ และทักษะที่เกี่ยวข้องในแต่ละสายอาชีพได้ถูกต้อง & แสดงรายละเอียดของอาชีพ และทักษะที่เกี่ยวข้องในแต่ละสายอาชีพได้ถูกต้อง & ผ่าน   &            \\ \cline{2-6}
                             & ขยายและยุบทักษะของแต่ละสายอาชีพได้                     & สามารถขยายและยุบทักษะของแต่ละสายอาชีพได้ถูกต้อง                 & สามารถขยายและยุบทักษะของแต่ละสายอาชีพได้ถูกต้อง                 & ผ่าน   & มีทักษะที่ชิดกัน \\\hline
              \caption{ตารางข้อมูลการทดสอบระบบ Career Exploration}
              \label{tbl:test-scenario-cexploration}
          \end{longtable}
    \item Google Login
          \begin{longtable}{|>{\raggedright\arraybackslash}p{0.1\linewidth}|>{\raggedright\arraybackslash}p{0.15\linewidth}|>{\raggedright\arraybackslash}p{0.17\linewidth}|>{\raggedright\arraybackslash}p{0.17\linewidth}|>{\centering}p{0.1\linewidth}|>{\raggedright\arraybackslash}p{0.1\linewidth}|} \hline
              หัวข้อ       & กรณีทดสอบ        & ผลที่คาดว่าจะได้รับ                                    & ผลที่ได้รับ                                           & ผลลัพธ์ & หมายเหตุ \\ \hline
              \endhead
              เข้าสู่ระบบ   & เข้าสู่ระบบสมาชิก   & สามารถเข้าสู่ระบบได้                                  & สามารถเข้าสู่ระบบได้                                  & ผ่าน   &         \\ \cline{2-6}
                         & ประวัติการทำนาย    & ประวัติการทำนายของผู้ใช้งานถูกจัดเก็บไว้                    & ประวัติการทำนายของผู้ใช้งานถูกจัดเก็บไว้                    & ผ่าน   &         \\\hline
              ออกจากระบบ & ออกจากระบบสมาชิก & ออกจากระบบสมาชิก                                   & ออกจากระบบสมาชิก                                   & ผ่าน   &         \\ \cline{2-6}
                         & ประวัติการทำนาย    & ผลที่ทำนายระหว่างเข้าสู่ระบบจะไม่ถูกแสดงหลังออกจากระบบสมาชิก & ผลที่ทำนายระหว่างเข้าสู่ระบบจะไม่ถูกแสดงหลังออกจากระบบสมาชิก & ผ่าน   &         \\\hline
              \caption{ตารางข้อมูลการทดสอบระบบ Google Login}
              \label{tbl:test-scenario-googlelogin}
          \end{longtable}
\end{enumerate}

% \begin{longtable}{|>{\centering}p{0.1\linewidth}|p{0.1\linewidth}|p{0.1\linewidth}|p{0.15\linewidth}|p{0.14\linewidth}|>{\centering}p{0.05\linewidth}|p{0.1\linewidth}|} \hline
%     % \caption{ตารางข้อมูลการทดสอบระบบ}\label{tbl:test-scenario}
%     % \resizebox{\textwidth}{!}{ %
%     % \begin{tabular}{c|l|l|l|l|c} \hline
%     % \begin{tabularx}{\textwidth}{X|X|X|X|X|c|X}
%     หัวข้อ               & รายละเอียด                        & กรณีทดสอบ                                          & ผลที่คาดว่าจะได้รับ                                                              & ผลที่ได้รับ                                                                            & ผลลัพธ์ & หมายเหตุ                                                \\ \hline
%     \endhead
%     Career Prediction  & ตรวจสอบการทำงานของระบบทำนายสายอาชีพ & กรอกรายละเอียดครบตามช่องที่กำหนด                     & สามารถทำนายสายอาชีพได้                                                         & สามารถทำนายสายอาชีพได้                                                                & ผ่าน      &                                                        \\ \cline{3-7}
%                        &                                  & กรอกรายละเอียดไม่ครบตามช่องที่กำหนด                       & ไม่สามารถทำนายสายอาชีพได้                                                       & ไม่สามารถทำนายสายอาชีพได้                                                              & ผ่าน      &                                                        \\
%     \cline{2-7}
%                        & ระบบจัดการผลการทำนายอาชีพ           & บันทึกผลการทำนายสายอาชีพ                              & สามารถบันทึกผลการทำนายสายอาชีพได้                                                & สามารถบันทึกผลการทำนายสายอาชีพได้                                                       & ผ่าน      &                                                        \\ \cline{3-7}
%                        &                                  & แก้ไขผลการทำนายสายอาชีพ                              & สามารถแก้ไขผลการทำนายสายอาชีพได้                                                & สามารถแก้ไขผลการทำนายสายอาชีพได้                                                       & ผ่าน      &                                                        \\ \cline{3-7}
%                        &                                  & ยกเลิกผลการทำนายสายอาชีพ                             & สามารถยกเลิกผลการทำนายสายอาชีพได้                                               & สามารถยกเลิกผลการทำนายสายอาชีพได้                                                      & ผ่าน      &                                                        \\ \cline{3-7}
%                        &                                  & อ่านรายละเอียดเชิงลึกของสายอาชีพที่ทำนายได้                & สามารถอ่านรายละเอียดเชิงลึกของสายอาชีพที่ทำนายได้                                    & สามารถอ่านรายละเอียดเชิงลึกของสายอาชีพที่ทำนายได้                                           & ผ่าน      &                                                        \\
%     \cline{2-7}
%                        & ระบบการ์ดที่ทำนายไว้                  & รายละเอียดโดยย่อของประวัติการทำนาย                     & แสดงรายละเอียดโดยย่อของประวัติการทำนายได้ถูกต้อง                                    & แสดงรายละเอียดโดยย่อของประวัติการทำนายได้ถูกต้อง                                           & ผ่าน      &                                                        \\ \cline{3-7}
%                        &                                  & อ่านรายละเอียดเชิงลึกของสายอาชีพที่เคยทำนายไว้             & สามารถอ่านรายละเอียดเชิงลึกของสายอาชีพที่เคยทำนายไว้                                 & สามารถอ่านรายละเอียดเชิงลึกของสายอาชีพที่เคยทำนายไว้                                        & ผ่าน      &                                                        \\ \cline{3-7}
%                        &                                  & ลบประวัติที่เคยทำนายไว้                                 & สามารถลบประวัติที่เคยทำนายไว้                                                     & สามารถลบประวัติที่เคยทำนายไว้                                                            & ผ่าน      &                                                        \\ \hline
%     Career Insight     & รายละเอียดเชิงลึกของสายอาชีพ         & รายละเอียดที่แสดงตรงตามกับสายอาชีพที่ถูกทำนายได้            & รายละเอียดตรงกับสายอาชีพที่ทำนายได้                                                & รายละเอียดตรงกับสายอาชีพที่ทำนายได้                                                       & ผ่าน      &                               \\ \cline{3-7}
%                        &                                  & รายละเอียดที่แสดงมีความถูกต้อง ตรงกับในฐานข้อมูล           & รายละเอียดตรงกับในฐานข้อมูล                                                     & รายละเอียดตรงกับในฐานข้อมูล                                                            & ผ่าน      &                                                        \\ \cline{3-7}
%                        &                                  & สลับอาชีพที่เกี่ยวข้องกับสายอาชีพที่ทำนายได้                   & รายละเอียดของอาชีพถูกสับเปลี่ยนตามที่เลือก                                           & รายละเอียดของอาชีพถูกสับเปลี่ยนตามที่เลือก                                                  & ผ่าน      &                                                        \\ \cline{3-7}
%                        &                                  & รายละเอียดทักษะที่เหมาะสม, ทักษะที่ควรเรียนรู้, ทักษะอื่น ๆ    & แสดงรายละเอียดทักษะที่เหมาะสม, ทักษะที่ควรเรียนรู้, ทักษะอื่น ๆ ได้ถูกต้องตามรายละเอียดที่กรอก & แสดงรายละเอียดทักษะที่เหมาะสม, ทักษะที่ควรเรียนรู้, ทักษะอื่น ๆ ไม่ได้ถูกต้องตามรายละเอียดที่กรอกทั้งหมด & ไม่ผ่าน    & ไม่สามารถจับคู่กับคำที่พ้องความหมายได้ เช่น Automated กับ Automate \\
%     \cline{2-7}
%                        & เลือกการ์ดทำนาย                     & เลือกประวัติการทำนายที่เคยทำนายไปแล้ว                     & สามารถเลือกประวัติการทำนายที่เคยทำนายไปแล้วได้                                       & สามารถเลือกประวัติการทำนายที่เคยทำนายไปแล้วได้                                              & ผ่าน      &                                                        \\
%     \cline{2-7}
%                        & อาชีพและทักษะที่เกี่ยวข้อง              & แสดงภาพรวมของอาชีพ และทักษะที่เกี่ยวข้องอื่น ๆ             & สามารถดูภาพรวมของอาชีพ และทักษะที่เกี่ยวข้องอื่น ๆ ได้ ในหน้าของสายใยอาชีพได้             & สามารถดูภาพรวมของอาชีพ และทักษะที่เกี่ยวข้องอื่น ๆ ได้ ในหน้าของสายใยอาชีพได้                    & ผ่าน      &                                                        \\ \hline
%     Career Exploration & แผนผังสายใยอาชีพ                   & แสดงรายละเอียดของอาชีพ และทักษะที่เกี่ยวข้องในแต่ละสายอาชีพ & แสดงรายละเอียดของอาชีพ และทักษะที่เกี่ยวข้องในแต่ละสายอาชีพได้ถูกต้อง                    & แสดงรายละเอียดของอาชีพ และทักษะที่เกี่ยวข้องในแต่ละสายอาชีพได้ถูกต้อง                           & ผ่าน      &                                                        \\ \cline{3-7}
%                        &                                  & ขยายและยุบทักษะของแต่ละสายอาชีพได้                     & สามารถขยายและยุบทักษะของแต่ละสายอาชีพได้ถูกต้อง                                    & สามารถขยายและยุบทักษะของแต่ละสายอาชีพได้ถูกต้อง                                           & ผ่าน      & มีทักษะที่ชิดกัน เมื่อขยายทักษะ                            \\\hline
%     % \end{tabularx}
%     % \end{tabular} %
%     % }
%     \caption{ตารางข้อมูลการทดสอบระบบ}
%     \label{tbl:test-scenario}
% \end{longtable}
% \end{table}
\section{Model Result}
หลังจากที่คณะผู้จัดทำงานได้\hyperref[subsec:Data Collecting]{รวบรวมข้อมูล} และ\hyperref[subsec:Data Preparation]{เตรียมข้อมูล}ทางคณะผู้จัดทำจึงทำดำเนินขั้นตอนต่าง ๆ ดังนี้
\subsection{Preprocessing Data}
ผู้จัดทำได้สังเกตเห็นว่ามีเรซูเมที่ซ้ำกัน ทางผู้จัดทำจึงได้ทำการลบข้อมูลที่มีลักษณะเหมือนกันออก จนเหลือทั้งหมด 205 เล่ม
ทางคณะผู้จัดทำได้นำข้อมูลในเรซูเมมาผ่านกระบวนการต่าง ๆ ได้แก่
\begin{enumerate}
    \item ลบอักขระพิเศษ (Remove Punctation) : คณะผู้จัดทำจะทำการลบ ลิงก์ (URL), การกล่าวถึง (@), อักขระพิเศษ (Punctation), เว้นวรรคที่มากเกิน (Extra Whitespace)
    \item การแปรหรือผันคำ (Lemmatize Text) : คณะผู้จัดทำจะทำการแปลงคำให้กลับไปอยู่ในรูปเดิมของคำ เช่น Connects -> Connect หรือ Building -> Build
    \item ลบคำบางคำออก (Remove Specific Word) : คณะผู้จัดทำได้ลบคำที่พบได้บ่อยในชุดข้อมูล และคิดว่าไม่ได้สำคัญในการทำนาย เช่น using, Ltd, Maharashtra เป็นต้น
\end{enumerate}
\subsection{Train Test Splitting}
คณะผู้จัดทำได้ทำการแบ่งชุดข้อมูลออกเป็น 2 ส่วน ข้อมูลสำหรับการทดลอง ข้อมูลสำหรับการประเมินโมเดล
และคณะผู้จัดทำได้รวบรวมเรซูเมจากนักศึกษาวิศวกรรมคอมพิวเตอร์ มจธ. จำนวนดังตารางต่อไปนี้
\begin{table}[H]
    \begin{tabularx}{\textwidth}{|X|X|l|} \hline
        ประเภทของชุดข้อมูล & จำนวนเรซูเม            & หมายเหตุ                                   \\ \hline
        Training Set    & 164 (80\% ของชุดข้อมูล) &                                           \\ \hline
        Validation Set  & 41 (20\% ของชุดข้อมูล)  &                                           \\ \hline
        Testing Set     & 13                   & เรซูเมจากนักศึกษาวิศวกรรมคอมพิวเตอร์ มจธ. รุ่น 34 \\ \hline
    \end{tabularx}
\end{table}
\subsection{Model Experiment}
คณะผู้จัดทำได้นำ Machine Learning ต่าง ๆ มาทดลองทั้งหมด 6 ประเภท เนื่องจากทางคณะผู้จัดทำได้ทำการศึกษาลักษณะของชุดข้อมูล ทำให้เห็นว่าเรซูเมส่วนใหญ่จะมีลักษณะเป็นหัวข้อ ไม่ได้เป็นประโยคยาวที่ข้อความจะสัมพันธ์กัน
ทางคณะผู้จัดจึงคิดว่า \hyperref[subsec:lstm]{Long Short Term Memory (LSTM)} ที่เน้นทำความเข้าใจความสัมพันธ์ของประโยค
อาจจะไม่เหมาะกับชุดข้อมูลนี้ และมีผลการทดลองความแม่นยำดังนี้
\begin{table}[H]
    \caption{ตารางเปรียบเทียบความแม่นยำของแต่ละโมเดล}
    \label{tab:Model accuracy}
    \begin{tabularx}{\textwidth}{|X|>{\centering\arraybackslash}X|>{\centering\arraybackslash}X|} \hline
        \multirow{2}{*}{Machine Learning Algorithm} & \multicolumn{2}{c|}{Accuracy (Resumes, Percent)}               \\ \cline{2-3}
                                                    & \centering Validation Set                       & Testing Set \\ \hline
        Multinomial Naive Bayes                     & 19 (46.34\%)                                    & 2 (15.38\%) \\ \hline
        Gaussian Naive Bayes                        & 27 (65.85\%)                                    & 3 (23.08\%) \\ \hline
        K-Neighbors                                 & 37 (90.24\%)                                    & 9 (69.23\%) \\ \hline
        OneVsRest (+ K-Neighbors)                   & 37 (90.24\%)                                    & 9 (69.23\%) \\ \hline
        Support Vector Machine                      & 36 (87.80\%)                                    & 7 (53.85\%) \\ \hline
        Logistic Regression                         & 35 (85.37\%)                                    & 7 (53.85\%) \\ \hline
        Bidirectional LSTM                          & 22 (54.84\%)                                    & 4 (30.70\%) \\ \hline
    \end{tabularx}
\end{table}
และเนื่่องจาก K-Neighbors มีความแม่นยำที่สูงที่สุด คณะผู้จัดทำจึงอยากที่จะแสดงผล Confusion Matrix เพื่อตรวจสอบว่าความผิดพลาดของโมเดลจะทำนายที่อาชีพไหน โดยที่หลักจะแสดงอาชีพที่โมเดลทำนาย และแถวจะแสดงเป็นอาชีพที่ถูกต้อง
\begin{table}[H]
    \caption{ตาราง Confusion Matrix ของ K-Nearest Neighbors ในการทำนาย Validation Set}
    \label{tab:confusion matrix}
    \begin{tabularx}{\textwidth}{|>{\centering\arraybackslash}X|>{\centering\arraybackslash}X|>{\centering\arraybackslash}X|>{\centering\arraybackslash}X|>{\centering\arraybackslash}X|>{\centering\arraybackslash}X|>{\centering\arraybackslash}X|} \hline
                 & Cloud                    & Data                     & Designer                 & Dev                       & QA                       & Security                 \\ \hline
        Cloud    & {\cellcolor[gray]{.9}} 5 & 0                        & 0                        & 0                         & 0                        & 0                        \\ \hline
        Data     & 1                        & {\cellcolor[gray]{.9}} 6 & 0                        & 0                         & 0                        & 0                        \\ \hline
        Designer & 0                        & 0                        & {\cellcolor[gray]{.9}} 2 & 0                         & 0                        & 0                        \\ \hline
        Dev      & 1                        & 0                        & 1                        & {\cellcolor[gray]{.9}} 13 & 1                        & 0                        \\ \hline
        QA       & 0                        & 0                        & 0                        & 0                         & {\cellcolor[gray]{.9}} 6 & 0                        \\ \hline
        Security & 0                        & 0                        & 0                        & 0                         & 0                        & {\cellcolor[gray]{.9}} 5 \\ \hline
        \multicolumn{7}{l}{\textbf{หมายเหตุ} : ชื่อแถวกับหลักเป็นคำย่อของ Cloud Management, Data \& AI, Designer, Developer, QA \& Tester, Security}                                        \\ \hline \hline
    \end{tabularx}
\end{table}
จะพบว่าโมเดลที่มีแนวโน้มที่จะทำนายเป็น Developer ที่ค่อนข้างสูง อาจจะเนื่องจากที่ชุดข้อมูลมีจำนวนอาชีพของ Developer ที่เยอะ
\subsection{Model Evaluation}
คณะผู้จัดทำคาดหวังจะนำผลลัพธ์ที่ได้จาก Model ไปเปรียบเทียบกับความคิดเห็น
ของผู้ที่มีประสบการณ์อยู่ภายในสายงานนี้ โดยอาจจะเป็นอาจารย์ภาควิชาวิศวกรรมคอมพิวเตอร์
หรือผู้ที่ทำงานอยู่ในแผนกบุคคลของบริษัทเทคโนโลยี เพื่อนำมาเปรียบเทียบกับ Test Set ของเราต่อไป

\emph{จัดทำในอนาคต}


%%%%%%%%%%%%%%%%%%%%%%%%%%%%%%%%%%%%%%%%%%%%%%%%%%%%%%%%%%%%%%%
%%%%%%%%%%%%%%%%%%%% Conclusions %%%%%%%%%%%%%%%%%%%%%%%%%%%%%
%%%%%%%%%%%%%%%%%%%%%%%%%%%%%%%%%%%%%%%%%%%%%%%%%%%%%%%%%%%%%%%

\chapter{บทสรุป}

\section{สรุปผลเบื้องต้นของโครงงาน}
\par{ภาพรวมของโครงงาน...}

\section{ปัญหาที่พบและการแก้ไข}
\subsection{การขาดปริมาณข้อมูล}
ทางคณะผู้จัดทำ พบปัญหากับปริมาณข้อมูลเรซูเมมีปริมาณที่ค่อนข้างน้อย\cite{dataset} 
เนื่องจากเป็นข้อมูลที่ละเอียดอ่อน และชุดข้อมูลยังค่อนข้างเก่าตั้งแต่ปีพุทธศักราช 2564 ทำให้ยากที่จะสามารถรวบรวมให้ได้ในปริมาณที่มากได้ ทำให้โมเดลยังรู้จักคำที่ไม่มาก และมักเป็นเครื่องมือที่ค่อนข้างเก่า
คณะผู้จัดทำจึงทำการแก้ไขปัญหานี้ด้วยการ
\begin{itemize}
    \item \textbf{รวบรวมข้อมูลเรซูเมจากผู้ใช้งานจริง} : คณะผู้จัดทำจะนำข้อมูลที่ผู้ใช้งานกรอกในการทำนายเรซูเมมาใช้เป็นชุดข้อมูลเพิ่มเติม โดยที่คณะผู้จัดทำ
จะขอความร่วมมือจากบุคคลที่ทำงานอยู่ในงานฝ่ายบุคคลของบริษัทเทคโนโลยีที่น่าเชื่อถือ หรืออาจารย์ภาควิชาวิศวกรรมคอมพิวเตอร์ มจธ. ในการระบุสายอาชีพจากเรซูเมนั้น
    \item \textbf{รวบรวมข้อมูลอื่นเพิ่มเติม} : คณะผู้จัดทำจะนำข้อมูลประกาศรับสมัครพนักงานของบริษัทเทคโนโลยีชั้นนำมารวบรวมเป็นข้อมูลสำหรับการปรับปรุงโมเดล
โดยที่คณะผู้จัดทำคาดหวังว่ารายละเอียดของการรับสมัครพนักงานเหล่านั้น จะทำให้โมเดลรู้จักคำสำคัญของแต่ละสายอาชีพ และช่วยให้การทำนายมีประสิทธิภาพมากยิ่งขึ้น
\end{itemize}
\subsection{ข้อจำกัดเรื่องการใช้งานเครื่องมือเพิ่มเติม ซึ่งต้องเสียค่าใช้จ่าย}
\par{แนวทางการแก้ไข....}

\section{แนวทางการพัฒนาต่อไปในอนาคต}
\subsection{Web Scraping เพื่อการแนะนำช่องทางการเรียนรู้}
\par{แนวทางการพัฒนา จะนำมาใช้อย่างไร เพื่ออะไร....}

%%%%%%%%%%%%%%%%%%%%%%%%%%%%%%%%%%%%%%%%%%%%%%%%%%%%%%%%%%%%%%%
%%%%%%%%%%%%%%%%%%%% Bibliography %%%%%%%%%%%%%%%%%%%%%%%%%%%%%
%%%%%%%%%%%%%%%%%%%%%%%%%%%%%%%%%%%%%%%%%%%%%%%%%%%%%%%%%%%%%%%

%%%% Comment this in your report to show only references you have
%%%% cited. Otherwise, all the references below will be shown.
%\nocite{*}
%% Use the kmutt.bst for bibtex bibliography style 
%% You must have cpe.bib and string.bib in your current directory.
%% You may go to file .bbl to manually edit the bib items.

% Sept, 2021 by Thanin
% improve url breaks to prevent unnecessary big white spaces in some cases
\makeatletter
\g@addto@macro{\UrlBreaks}{\UrlOrds}
\makeatother
% 

\bibliographystyle{kmutt}
\bibliography{string,cpe}

%%%%%%%%%%%%%%%%%%%%%%%%%%%%%%%%%%%%%%%%%%%%%%%%%%%%%%%%%%%%%%%
%%%%%%%%%%%%%%%%%%%%%%%% Appendix %%%%%%%%%%%%%%%%%%%%%%%%%%%%%
%%%%%%%%%%%%%%%%%%%%%%%%%%%%%%%%%%%%%%%%%%%%%%%%%%%%%%%%%%%%%%%
% \appendix{ชื่อภาคผนวกที่ 1}
% \noindent{\large\bf ใส่หัวข้อตามความเหมาะสม} \\

% This is where you put hardware circuit diagrams, detailed experimental data in tables or source codes, etc.. \\ \bigskip


% \begin{figure}[!h]
%     \caption{This is the figure x11 ทดสอบ จาก \href{https://www.google.com} {https://www.google.com}}\label{fig:x1}
% \end{figure}


% This appendix describes two static allocation methods for fGn (or fBm)
% traffic. Here, $\lambda$ and $C$ are respectively the traffic arrival
% rate and the service rate per dimensionless time step. Their unit are
% converted to a physical time unit by multiplying the step size
% $\Delta$. For a fBm self-similar traffic source,
% Norros~\cite{norros95} provides its EB as
% \begin{equation}\label{eq:norros}
%     C = \lambda + (\kappa(H)\sqrt{-2\ln\epsilon})^{1/H}a^{1/(2H)}x^{-(1-H)/H}\lambda^{1/(2H)}
% \end{equation}
% where $\kappa(H) = H^H(1-H)^{(1-H)}$. Simplicity in the calculation is
% the attractive feature of (\ref{eq:norros}).

% The MVA technique developed in~\cite{kim01} so far provides the most
% accurate estimation of the loss probability compared to previous
% bandwidth allocation techniques according to simulation results.
% Consider a discrete-time queueing system with constant service rate
% $C$ and input process $\lambda_n$ with $\mathbb{E}\{\lambda_n\} =
%     \lambda$ and $\mathrm{Var}\{\lambda_n\} = \sigma^2$.  Define $X_n \equiv
%     \sum_{k=1}^n \lambda_k - Cn$.  The loss probability due to the MVA
% approach is given by
% \begin{equation}\label{eq:loss_mva}
%     \varepsilon \approx \alpha e^{-m_x/2}
% \end{equation}
% where
% \begin{equation}\label{eq:mx}
%     m_x = \min_{n \geq 0} \frac{((C-\lambda)n + B)^2}{\mathrm{Var}\{X_n\}} =
%     \frac{((C-\lambda)n^\ast + B)^2}{\mathrm{Var}\{X_{n^{\ast}}\}}
% \end{equation}
% and
% \begin{equation}\label{eq:alpha}
%     \alpha =
%     \frac{1}{\lambda\sqrt{2\pi\sigma^2}}\exp\left(\frac{(C-\lambda)^2}{2\sigma^2}\right)
%     \int_C^\infty (r-C)\exp\left(\frac{(r-\lambda)^2}{2\sigma^2}\right)\, dr
% \end{equation}
% For a given $\varepsilon$, we numerically solve for $C$ that satisfies
% (\ref{eq:loss_mva}). Any search algorithm can be used to do the task.
% Here, the bisection method is used.

% Next, we show how $\mathrm{Var}\{X_n\}$ can be determined.  Let
% $C_{\lambda}(l)$ be the autocovariance function of $\lambda_n$.  The
% MVA technique basically approximates the input process $\lambda_n$
% with a Gaussian process, which allows $\mathrm{Var}\{X_n\}$ to be
% represented by the autocovariance function.  In particular, the
% variance of $X_n$ can be expressed in terms of $C_{\lambda}(l)$ as
% \begin{equation}
%     \mathrm{Var}\{X_n\} = nC_{\lambda}(0) + 2\sum_{l=1}^{n-1} (n-l)C_{\lambda}(l)
% \end{equation}
% Therefore, $C_{\lambda}(l)$ must be known in the MVA technique, either
% by assuming specific traffic models or by off-line analysis in case of
% traces.  In most practical situations, $C_{\lambda}(l)$ will not be
% known in advance, and an on-line measurement algorithm developed
% in~\cite{eun01} is required to jointly determine both $n^\ast$ and
% $m_x$. For fGn traffic, $\mathrm{Var}\{X_n\}$ is equal to $\sigma^2
%     n^{2H}$, where $\sigma^2 = \mathrm{Var}\{\lambda_n\}$, and we can find
% the $n^\ast$ that minimizes (\ref{eq:mx}) directly. Although $\lambda$
% can be easily measured, it is not the case for $\sigma^2$ and $H$.
% Consequently, the MVA technique suffers from the need of prior
% knowledge traffic parameters.


%%%%%%%%%%%%%%%%%%%%%%%%%%%%%%%%%%%%%%%%%%%%%%%%%%%%%%%%%%
%%%%%%%%%%%%%%% The 2nd appendix %%%%%%%%%%%%%%%%%%%%%%%%%%
%%%%%%%%%%%%%%%%%%%%%%%%%%%%%%%%%%%%%%%%%%%%%%%%%%%%%%%%%%
% \appendix{ชื่อภาคผนวกที่ 2}
% \noindent{\large\bf ใส่หัวข้อตามความเหมาะสม} \\


% \begin{figure}[!h]
%     \caption{This is the figure x11 ทดสอบ จาก \href{https://www.google.com} {https://www.google.com}}\label{fig:x1}
% \end{figure}

% Next, we show how $\mathrm{Var}\{X_n\}$ can be determined.  Let
% $C_{\lambda}(l)$ be the autocovariance function of $\lambda_n$.  The
% MVA technique basically approximates the input process $\lambda_n$
% with a Gaussian process, which allows $\mathrm{Var}\{X_n\}$ to be
% represented by the autocovariance function.  In particular, the
% variance of $X_n$ can be expressed in terms of $C_{\lambda}(l)$ as
% \begin{equation}
%     \mathrm{Var}\{X_n\} = nC_{\lambda}(0) + 2\sum_{l=1}^{n-1} (n-l)C_{\lambda}(l)
% \end{equation}

% \noindent{\large\bf Add more topic as you need} \\

% Therefore, $C_{\lambda}(l)$ must be known in the MVA technique, either
% by assuming specific traffic models or by off-line analysis in case of
% traces.  In most practical situations, $C_{\lambda}(l)$ will not be
% known in advance, and an on-line measurement algorithm developed
% in~\cite{eun01} is required to jointly determine both $n^\ast$ and
% $m_x$. For fGn traffic, $\mathrm{Var}\{X_n\}$ is equal to $\sigma^2
%     n^{2H}$, where $\sigma^2 = \mathrm{Var}\{\lambda_n\}$, and we can find
% the $n^\ast$ that minimizes (\ref{eq:mx}) directly. Although $\lambda$
% can be easily measured, it is not the case for $\sigma^2$ and $H$.
% Consequently, the MVA technique suffers from the need of prior
% knowledge traffic parameters.



\end{document}
\section{Model Result}
หลังจากที่คณะผู้จัดทำงานได้\hyperref[subsec:Data Collecting]{รวบรวมข้อมูล} และ\hyperref[subsec:Data Preparation]{เตรียมข้อมูล}ทางคณะผู้จัดทำจึงทำดำเนินขั้นตอนต่าง ๆ ดังนี้
\subsection{Preprocessing Data}
ผู้จัดทำได้สังเกตเห็นว่ามีเรซูเมที่ซ้ำกัน ทางผู้จัดทำจึงได้ทำการลบข้อมูลที่มีลักษณะเหมือนกันออก จนเหลือทั้งหมด 205 เล่ม
ทางคณะผู้จัดทำได้นำข้อมูลในเรซูเมมาผ่านกระบวนการต่าง ๆ ได้แก่
\begin{enumerate}
    \item ลบอักขระพิเศษ (Remove Punctation) : คณะผู้จัดทำจะทำการลบ ลิงก์ (URL), การกล่าวถึง (@), อักขระพิเศษ (Punctation), เว้นวรรคที่มากเกิน (Extra Whitespace)
    \item การแปรหรือผันคำ (Lemmatize Text) : คณะผู้จัดทำจะทำการแปลงคำให้กลับไปอยู่ในรูปเดิมของคำ เช่น จาก Connects เป็น Connect หรือ จาก Building เป็น Build
    \item ลบคำบางคำออก (Remove Specific Word) : คณะผู้จัดทำได้ลบคำที่พบได้บ่อยในชุดข้อมูล และคิดว่าไม่ได้สำคัญในการทำนาย เช่น using, Ltd, Maharashtra เป็นต้น
\end{enumerate}
\subsection{Train Test Splitting}
คณะผู้จัดทำได้ทำการแบ่งชุดข้อมูลออกเป็น 2 ส่วน ข้อมูลสำหรับการทดลอง ข้อมูลสำหรับการประเมินโมเดล
และคณะผู้จัดทำได้รวบรวมเรซูเมจากนักศึกษาวิศวกรรมคอมพิวเตอร์ มจธ.
\begin{table}[H]
    \caption{ตารางแสดงจำนวนของการแบ่งชุดข้อมูล}
    \label{tbl:train-test-split}
    \begin{tabularx}{\textwidth}{|X|X|l|} \hline
        ประเภทของชุดข้อมูล & จำนวนเรซูเม            & หมายเหตุ                                   \\ \hline
        Training Set    & 164 (80\% ของชุดข้อมูล) &                                           \\ \hline
        Validation Set  & 41 (20\% ของชุดข้อมูล)  &                                           \\ \hline
        Testing Set     & 14                   & เรซูเมจากนักศึกษาวิศวกรรมคอมพิวเตอร์ มจธ. รุ่น 34 \\ \hline
    \end{tabularx}
\end{table}
\subsection{Model Experiment}
หลังจากคณะผู้จัดทำได้นำ Machine Learning ต่าง ๆ และ Bi-LSTM ซึ่งเป็น Deep Learning มาทดลองรวมครบ 7 ประเภทแล้ว จึงได้รับผลการทดลองความแม่นยำดังนี้
\begin{table}[H]
    \caption{ตารางเปรียบเทียบความแม่นยำของแต่ละโมเดลกับ Validation Set}
    \label{tab:Model accuracy}
    \begin{tabularx}{\textwidth}{|X|>{\centering\arraybackslash}X|} \hline
        Machine Learning Algorithm & Accuracy (Resumes, Percent) \\ \hline
        Multinomial Naive Bayes                     & 19 (46.34\%)                \\ \hline
        Gaussian Naive Bayes                        & 27 (65.85\%)                \\ \hline
        K-Nearest Neighbors                                 & 37 (90.24\%)                \\ \hline
        OneVsRest (+ K-Nearest Neighbors)                   & 37 (90.24\%)                \\ \hline
        Support Vector Machine                      & 36 (87.80\%)                \\ \hline
        Logistic Regression                         & 35 (85.37\%)                \\ \hline
        Bidirectional LSTM                          & 22 (54.84\%)                \\ \hline
    \end{tabularx}
\end{table}

\par{
    สังเกตได้ว่า \textbf{โมเดล Bi-LSTM ไม่ได้ให้ผลลัพธ์ที่ดีอย่างที่คาดการณ์และคาดหวังไว้} อย่างที่ทางคณะผู้จัดทำคิด โดยจากการศึกษาลักษณะของชุดข้อมูล และค้นหาสาเหตุ
    ทางเราคาดการณ์ว่าเป็นเพราะเรซูเมส่วนใหญ่จะมีลักษณะเป็นหัวข้อ วลีของคำศัพท์ต่าง ๆ หรือศัพท์เทคนิคโดยตรง ไม่ได้เป็นประโยคยาวที่ข้อความจะสัมพันธ์กัน
    โดยหัวข้อที่มีลักษณะเป็นเช่นนั้น มีเพียงหัวข้อเกี่ยวกับประสบการณ์การทำงาน ซึ่งมีความยาวเป็นเพียงหนึ่งในสามของข้อมูลแต่ละชุดเท่านั้น
    ทางคณะผู้จัดทำจึงคิดว่า \hyperref[subsec:lstm]{Long Short Term Memory (LSTM)} ที่เน้นทำความเข้าใจความสัมพันธ์ของประโยค
    อาจจะไม่เหมาะกับชุดข้อมูลนี้ที่เรากำลังเทรน ส่งผลให้ประสิทธิภาพด้อยกว่าที่คาด
}
\par{
    จากการสังเกตประสิทธิภาพของ Bi-LSTM ที่ด้อยกว่าที่คาด เราจึงคาดการณ์อีกว่า สาเหตุที่โมเดลประเภท Machine learning ที่ใช้เทคนิค TF-IDF มีประสิทธิภาพที่ดีกว่า
    มีสาเหตุมาจากการที่ เรซูเมแต่ละสายอาชีพ จะมีคลังคำศัพท์เฉพาะทางที่เป็นลักษณะเฉพาะของสายอาชีพนั้น ๆ ผสมอยู่ จึงทำให้ TF-IDF ที่มองความถี่ของคำเป็นนัยยะสำคัญ
    สามารถจับความสำคัญของคำ และความสัมพันธ์ที่มีต่อสายอาชีพได้ดีกว่า อันเนื่องมาจากลักษณะของข้อมูลอย่างที่กล่าวไปข้างต้นในกรณีเดียวกับ Bi-LSTM
}

\par{
    เนื่องจาก K-Nearest Neighbors มีความแม่นยำที่สูงที่สุด และ ใช้ทรัพยากรน้อยกว่าเมื่อเทียบกับ OneVsRest คณะผู้จัดทำจึงเลือกที่จะพัฒนาโมเดลที่มีศักยภาพที่สุดนี้ต่อไปให้สุดทาง
    จึงต้องการที่จะแสดงผล Confusion Matrix เพื่อตรวจสอบว่าความผิดพลาดของโมเดลจะทำนายที่อาชีพไหน
    โดยที่หลักจะแสดงอาชีพที่โมเดลทำนาย และแถวจะแสดงเป็นอาชีพที่ถูกต้อง
}
\begin{table}[H]
    \caption{ตาราง Confusion Matrix ของ K-Nearest Neighbors ในการทำนาย Validation Set}
    \label{tab:confusion matrix}
    \begin{tabularx}{\textwidth}{|>{\centering\arraybackslash}X|>{\centering\arraybackslash}X|>{\centering\arraybackslash}X|>{\centering\arraybackslash}X|>{\centering\arraybackslash}X|>{\centering\arraybackslash}X|>{\centering\arraybackslash}X|} \hline
                 & Cloud                    & Data                     & Designer                 & Dev                       & QA                       & Security                 \\ \hline
        Cloud    & {\cellcolor[gray]{.9}} 5 & 0                        & 0                        & 0                         & 0                        & 0                        \\ \hline
        Data     & 1                        & {\cellcolor[gray]{.9}} 6 & 0                        & 0                         & 0                        & 0                        \\ \hline
        Designer & 0                        & 0                        & {\cellcolor[gray]{.9}} 2 & 0                         & 0                        & 0                        \\ \hline
        Dev      & 1                        & 0                        & 1                        & {\cellcolor[gray]{.9}} 13 & 1                        & 0                        \\ \hline
        QA       & 0                        & 0                        & 0                        & 0                         & {\cellcolor[gray]{.9}} 6 & 0                        \\ \hline
        Security & 0                        & 0                        & 0                        & 0                         & 0                        & {\cellcolor[gray]{.9}} 5 \\ \hline
        \multicolumn{7}{l}{\textbf{หมายเหตุ} : ชื่อแถวกับหลักเป็นคำย่อของ Cloud Management, Data \& AI, Designer, Developer, QA \& Tester, Security}                                        \\ \hline \hline
    \end{tabularx}
\end{table}
จะพบว่าโมเดลมีแนวโน้มที่จะทำนายเป็นสายอาชีพ Developer ค่อนข้างสูง ซึ่งอาจมาจากการที่ชุดข้อมูลมีจำนวนอาชีพของ Developer ที่เยอะมากกว่าสายอาชีพอื่น 
โมเดลจึงทำนายไปที่ Developer ซึ่งมีโอกาสถูกต้องสูงสุด สื่อให้เห็นว่า โมเดลจะสามารถพัฒนาต่อไปได้มากกว่านี้อย่างแน่นอนหากมีชุดข้อมูลมากกว่าปัจจุบัน
จึงเป็นส่วนหนึ่งของสาเหตุที่เราต้องการเก็บข้อมูลภายในเว็บแอปพลิเคชันเพิ่มเติม และนำมาพัฒนาต่อในอนาคตหากมีโอกาสอย่างที่เคยกล่าวไปใน\hyperref[subsec:Improvement Model]{การพัฒนาโมเดลต่อไป}